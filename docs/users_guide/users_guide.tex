\documentclass[12pt]{THG_Guide}

\title{\textbf{HDF5 DAOS VOL Connector User's Guide}}
\author{Jordan Henderson, Neil Fortner, Jerome Soumagne}
\date{March 29, 2019}

% For including the HDF image in the footer
%\usepackage{graphicx}

% For nicely-formatted tables
\usepackage{tabularx}

% To represent hyperlinks
\usepackage[colorlinks=true,linkcolor=blue,urlcolor=blue]{hyperref}

% To control the formatting of page headers
%\usepackage{fancyhdr}

% For coloring headers, footers and other elements according to typical THG document style
%\usepackage{xcolor}

% For correct page-numbering
%\usepackage{lastpage}

% To control the margins of the document
%\usepackage[margin=0.5in, top=0.7in, bottom=0.7in]{geometry}

%\pagestyle{fancy}
%\cfoot{}
%\renewcommand{\headrule}{\hbox to\headwidth{\color{blue}\leaders\hrule height \headrulewidth\hfill}}
%\fancyhead[LH]{\raisebox{-.7\baselineskip}[0pt][0pt]{February 28, 2019}}
%\fancyhead[RH]{\raisebox{-.7\baselineskip}[0pt][0pt]{THG 2019-02-28.v1}}
%\renewcommand{\footrule}{{\color{blue}\hrule}}
%\fancyfoot[LF]{\includegraphics{HDF_logo.jpg}}
%\fancyfoot[RF]{Page \thepage\ of \pageref*{LastPage}}
%\setlength{\parskip}{\baselineskip}
%\setlength{\parindent}{0ex}
\rfcversion{2019-03-29.v1}

\begin{document}

\maketitle

% Ensure header and footer get added to first page
%\thispagestyle{fancy}

%{\color{blue}\hrule}

%\begin{center}

%\begin{minipage}{.95\linewidth}
%% Abstract
\begin{abstract}
This document aims to be a helpful guide on how to use the HDF5 DAOS VOL connector to leverage the capabilities of the DAOS object storage system within an HDF5 application.
\end{abstract}

%\end{minipage}

%\end{center}

%\vspace{0.2in}
%{\color{blue}\hrule}

%\newpage

%\newpage

\tableofcontents
\newpage

%\section{Introduction}
%TBD
%\newpage

\section{Using the DAOS VOL connector with an HDF5 application}

This section outlines the unique aspects of writing, building and running HDF5 applications with the DAOS VOL connector.

\subsection{Writing HDF5 DAOS VOL connector applications}

Any HDF5 application using the DAOS VOL connector must:

\begin{enumerate}
    \item Include \texttt{daos\_vol\_public.h}, found in the \texttt{include} directory of the DAOS VOL connector installation directory.
    \item Link against \texttt{libhdf5\_vol\_daos.so} (or similar), found in the \texttt{lib} directory of the DAOS VOL connector installation directory.
\end{enumerate}

An HDF5 DAOS VOL connector application also requires three new function calls in addition to those for an equivalent HDF5 application (see Appendix~\ref{apdx:ref_manual} for more details):

\begin{itemize}
    \item \texttt{H5daos\_init()} - Initializes the DAOS VOL connector

    Called upon application startup, before any file is accessed.
    
    \item \texttt{H5Pset\_fapl\_daos()} - Set DAOS VOL connector access on File Access Property List.

    Called to prepare a FAPL to open a file through the DAOS VOL connector. See \href{https://support.hdfgroup.org/HDF5/Tutor/property.html#fa}{HDF5 File Access Property Lists} for more information about File Access Property Lists.

    \item \texttt{H5daos\_term()} - Cleanly shutdown the DAOS VOL connector

    Called on application shutdown, after all files have been closed.
\end{itemize}

\newpage

\subsubsection{Skeleton Example}

Below is a no-op application that opens and closes a file using the DAOS VOL connector. For clarity, no error-checking is performed.

\begin{verbatim}
#include "hdf5.h"
#include "daos_vol_public.h"

int main(void)
{
    hid_t fapl_id;
    hid_t file_id;

    H5daos_init(MPI_COMM_WORLD, pool_uuid, NULL);

    fapl_id = H5Pcreate(H5P_FILE_ACCESS);
    H5Pset_fapl_daos(fapl_id, MPI_COMM_WORLD, MPI_INFO_NULL);
    file_id = H5Fopen("my/file.h5");

    /* operate on file */

    H5Pclose(fapl_id);
    H5Fclose(file_id);

    H5daos_term();

    return 0;
}
\end{verbatim}

\newpage

\subsection{Building HDF5 DAOS VOL connector applications}

Assuming an HDF5 application has been written following the instructions in the previous section, the application must be built prior to running. In general, the application should be built as normal for any other HDF5 application.

To link in the required libraries, the compiler will likely require the additional linker flags:

\begin{verbatim}
-lhdf5_vol_daos -luuid
\end{verbatim}

However, these flags may vary depending on platform, compiler and installation location of the DAOS VOL connector. It is highly recommended that compilation of HDF5 DAOS VOL connector applications be done using the \texttt{h5cc/h5pcc} script, as it will manage linking with the HDF5 library. If HDF5 was built using Autotools, this script will be called \texttt{h5pcc} and may be found in the \texttt{bin} directory of the HDF5 installation. If HDF5 was built with CMake, this script will simply be called \texttt{h5cc} and can be found in the same location. The above notice about additional library linking applies to usage of \texttt{h5cc/h5pcc}. For example:

\begin{verbatim}
h5cc/h5pcc -lhdf5_vol_daos -luuid my_daosvol_application.c -o my_executable
\end{verbatim}

\newpage

\subsection{Running HDF5 DAOS VOL connector applications}

Running applications that use the HDF5 DAOS VOL connector requires access to a server which implements the \href{https://github.com/daos-stack/daos/blob/master/src/include/daos_api.h}{DAOS API}. Refer to \href{https://github.com/daos-stack/daos/blob/master/doc/quickstart.md}{DAOS Quick Start Guide} for more information on the setup process for this.

%\subsubsection{Starting the DAOS Server}
%\label{sec:daos_serv_start}
%TBD

\subsubsection{Runtime Environment}
\label{sec:runtime_env}

For the DAOS VOL connector to correctly interact with a DAOS API-aware server instance, there are two environment variables which should first be set. These are:

\begin{verbatim}
DAOS_POOL - The UUID of the DAOS pool to use

DAOS_SVCL - A comma-separated list of MPI ranks used for daos_pool_connect().
            Currently, for simple 1 MPI rank operations this should simply be
            specified as DAOS_SVCL=0
\end{verbatim}

\subsubsection{Example Applications}

The file \texttt{test/daos\_vol/test\_daos\_vol.c} contains several test functions that are examples of applications in miniature, focused on a particular behavior. These mini-applications test a moderate amount of HDF5's public API functionality with the DAOS VOL connector and should be a good indicator of whether the DAOS VOL connector is working correctly in conjunction with a running HDF5 DAOS API-aware instance.

In addition to this file, some of the example C applications included with HDF5 distributions have been adapted to work with the DAOS VOL connector and are included under the top-level \texttt{examples} directory in the DAOS VOL connector source root directory.

\newpage

\section{HDF5 Feature support}

\subsection{HDF5 API support}

The following sections serve to illustrate the DAOS VOL connector's support for the HDF5 API, as well as to highlight any differences between the expected behavior of an HDF5 API call versus the actual behavior as implemented by the VOL connector. If a particular HDF5 API call does not appear among these tables, it is most likely a native HDF5-specific API call which cannot be implemented by non-native HDF5 VOL connectors. These types of API calls are listed among the tables in Appendix~\ref{apdx:native_calls}.

\subsubsection{H5A interface}

\begin{center}

\textbf{Supported API calls}
\vspace{.2in} \\

\begin{tabularx}{\linewidth}{| X | X |}
\hline
% Center just the header row
\multicolumn{1}{| c |}{\textbf{API call}} & \multicolumn{1}{c |}{\textbf{Notes}} \\ \hline

H5Acreate(1/2) & \\ \hline
H5Acreate\_by\_name & \\ \hline
H5Aopen(\_by\_name) & \\ \hline
H5Aopen\_name & \\ \hline
H5Awrite & \\ \hline
H5Aread & \\ \hline
H5Aclose & \\ \hline
H5Aiterate(2) & \begin{itemize}
                    \item Restarting iteration from an index value is currently unsupported
                    \item Only \texttt{H5\_ITER\_NATIVE} is supported for the iteration order
                \end{itemize}\\ \hline
H5Aiterate\_by\_name & \begin{itemize}
                           \item Restarting iteration from an index value is currently unsupported
                           \item Only \texttt{H5\_ITER\_NATIVE} is supported for the iteration order
                       \end{itemize}\\ \hline
H5Aexists(\_by\_name) & \\ \hline
H5Aget\_name & \\ \hline
H5Aget\_space & \\ \hline
H5Aget\_type & \\ \hline
H5Aget\_create\_plist & \\ \hline

\end{tabularx}

\newpage

\textbf{Currently unsupported API calls}
\vspace{.1in} \\

\begin{tabularx}{\linewidth}{| X | X |}
\hline
% Center just the header row
\multicolumn{1}{| c |}{\textbf{API call}} & \multicolumn{1}{c |}{\textbf{Notes}} \\ \hline

H5Aopen\_by\_idx & Currently no support for attribute creation order \\ \hline
H5Aopen\_idx & Currently no support for attribute creation order \\ \hline
H5Arename(\_by\_name) & \\ \hline
H5Adelete(\_by\_name/\_by\_idx) & \\ \hline
H5Aget\_info(\_by\_name/\_by\_idx) & \\ \hline
H5Aget\_name\_by\_idx & Currently no support for attribute creation order \\ \hline
H5Aget\_storage\_size & \\ \hline

\end{tabularx}

\end{center}

\subsubsection{H5D interface}

\begin{center}

\textbf{Supported API calls}
\vspace{.2in} \\

\begin{tabularx}{\linewidth}{| X | X |}
\hline
% Center just the header row
\multicolumn{1}{| c |}{\textbf{API call}} & \multicolumn{1}{c |}{\textbf{Notes}} \\ \hline

H5Dcreate(1/2) & \\ \hline
H5Dcreate\_anon & \\ \hline
H5Dopen(1/2) & \\ \hline
H5Dwrite & \\ \hline
H5Dread & \\ \hline
H5Dclose & \\ \hline
H5Diterate & \\ \hline
H5Dget\_space & \\ \hline
H5Dget\_type & \\ \hline
H5Dget\_create\_plist & \\ \hline
H5Dget\_access\_plist & \\ \hline

\end{tabularx}

\textbf{Currently unsupported API calls}
\vspace{.2in} \\

\begin{tabularx}{\linewidth}{| X | X |}
\hline
% Center just the header row
\multicolumn{1}{| c |}{\textbf{API call}} & \multicolumn{1}{c |}{\textbf{Notes}} \\ \hline

H5Dflush & \\ \hline
H5Drefresh & \\ \hline
H5Dextend & \\ \hline
H5Dset\_extent & \\ \hline
H5Dget\_storage\_size & \\ \hline
H5Dget\_space\_status & Space status is currently always set to \texttt{H5D\_SPACE\_STATUS\_NOT\_ALLOCATED}\\ \hline
H5Dget\_offset & \\ \hline
H5Dvlen\_reclaim & \\ \hline
H5Dvlen\_get\_buf\_size & \\ \hline
H5Dscatter & \\ \hline
H5Dgather & \\ \hline
H5Dfill & \\ \hline

\end{tabularx}

\end{center}

\subsubsection{H5F interface}

\begin{center}

\textbf{Supported API calls}
\vspace{.2in} \\

\begin{tabularx}{\linewidth}{| X | X |}
\hline
% Center just the header row
\multicolumn{1}{| c |}{\textbf{API call}} & \multicolumn{1}{c |}{\textbf{Notes}} \\ \hline

H5Fcreate & \\ \hline
H5Fopen & \\ \hline
H5Freopen & \\ \hline
H5Fis\_accessible & \\ \hline
H5Fget\_create\_plist & \\ \hline
H5Fget\_access\_plist & \\ \hline
H5Fget\_intent & \\ \hline
H5Fget\_name & \\ \hline
H5Fget\_obj\_count & \\ \hline
H5Fget\_obj\_ids & \\ \hline
H5Fclose & \\ \hline

\end{tabularx}

\newpage

\textbf{Currently unsupported API calls}
\vspace{.2in} \\

\begin{tabularx}{\linewidth}{| X | X |}
\hline
% Center just the header row
\multicolumn{1}{| c |}{\textbf{API call}} & \multicolumn{1}{c |}{\textbf{Notes}} \\ \hline

H5Fflush & \\ \hline
H5Fmount & \\ \hline
H5Funmount & \\ \hline

\end{tabularx}

\end{center}

\newpage

\subsubsection{H5G interface}

\begin{center}

\textbf{Supported API calls}
\vspace{.2in} \\

\begin{tabularx}{\linewidth}{| X | X |}
\hline
% Center just the header row
\multicolumn{1}{| c |}{\textbf{API call}} & \multicolumn{1}{c |}{\textbf{Notes}} \\ \hline

H5Gcreate(1/2) & \\ \hline
H5Gcreate\_anon & \\ \hline
H5Gopen(1/2) & \\ \hline
H5Gclose & \\ \hline
H5Gget\_create\_plist & \\ \hline
H5Gget\_info(\_by\_name/\_by\_idx) & Of the four fields in the \texttt{H5G\_info\_t} struct:
                                     \begin{itemize}
                                         \item \texttt{storage\_type} is always set to \texttt{H5G\_STORAGE\_TYPE\_UNKNOWN}
                                         \item \texttt{nlinks} is set appropriately
                                         \item \texttt{max\_corder} is currently always set to 0
                                         \item \texttt{mounted} is currently always set to \texttt{FALSE}
                                     \end{itemize}
                                     \texttt{H5Gget\_info\_by\_idx} is currently unsupported due to the lack of support for link creation order
                                     \\ \hline
H5Gget\_num\_objs & \\ \hline
H5Glink(2) & Currently only soft link creation is supported\\ \hline

\end{tabularx}

\textbf{Currently unsupported API calls}
\vspace{.2in} \\

\begin{tabularx}{\linewidth}{| X | X |}
\hline
% Center just the header row
\multicolumn{1}{| c |}{\textbf{API call}} & \multicolumn{1}{c |}{\textbf{Notes}} \\ \hline

H5Gflush & \\ \hline
H5Grefresh & \\ \hline
H5Gmove(2) & Will be supported once \texttt{H5Lmove} is supported\\ \hline
H5Gunlink & Will be supported once \texttt{H5Ldelete} is supported\\ \hline
H5Gget\_linkval & Will be supported once \texttt{H5Lget\_val} is supported\\ \hline
H5Gget\_objname\_by\_idx & Will be supported once \texttt{H5Lget\_name\_by\_idx} is supported\\ \hline

\end{tabularx}

\end{center}

\newpage

\subsubsection{H5L interface}

\begin{center}

\textbf{Supported API calls}
\vspace{.2in} \\

\begin{tabularx}{\linewidth}{| X | X |}
\hline
% Center just the header row
\multicolumn{1}{| c |}{\textbf{API call}} & \multicolumn{1}{c |}{\textbf{Notes}} \\ \hline

H5Lcreate\_soft & \\ \hline
H5Lexists & \\ \hline
H5Literate(\_by\_name) & \begin{itemize}
                             \item Restarting iteration from an index value is currently unsupported
                             \item Only \texttt{H5\_ITER\_NATIVE} is supported for the iteration order
                         \end{itemize}\\ \hline

\end{tabularx}

\textbf{Currently unsupported API calls}
\vspace{.2in} \\

\begin{tabularx}{\linewidth}{| X | X |}
\hline
% Center just the header row
\multicolumn{1}{| c |}{\textbf{API call}} & \multicolumn{1}{c |}{\textbf{Notes}} \\ \hline

H5Lcopy & \\ \hline
H5Lmove & \\ \hline
H5Lcreate\_hard & \\ \hline
H5Lcreate\_external & \\ \hline
H5Lcreate\_ud & \\ \hline
H5Ldelete(\_by\_idx) & \\ \hline
H5Lget\_info(\_by\_idx) & \\ \hline
H5Lget\_name\_by\_idx & Currently no support for link creation order\\ \hline
H5Lget\_val(\_by\_idx) & \\ \hline
H5Lvisit(\_by\_name) & \\ \hline
H5Lregister & \\ \hline
H5Lunregister & \\ \hline
H5Lis\_registered & \\ \hline
H5Lunpack\_elink\_val & \\ \hline

\end{tabularx}

\end{center}

\newpage

\subsubsection{H5O interface}

\begin{center}

\textbf{Supported API calls}
\vspace{.2in} \\

\begin{tabularx}{\linewidth}{| X | X |}
\hline
% Center just the header row
\multicolumn{1}{| c |}{\textbf{API call}} & \multicolumn{1}{c |}{\textbf{Notes}} \\ \hline

H5Oopen & \\ \hline
H5Oopen\_by\_addr & \\ \hline
H5Oclose & \\ \hline

\end{tabularx}

\textbf{Currently unsupported API calls}
\vspace{.2in} \\

\begin{tabularx}{\linewidth}{| X | X |}
\hline
% Center just the header row
\multicolumn{1}{| c |}{\textbf{API call}} & \multicolumn{1}{c |}{\textbf{Notes}} \\ \hline

H5Oopen\_by\_idx & \\ \hline
H5Oexists\_by\_name & \\ \hline
H5Ovisit(1/2) & \\ \hline
H5Ovisit\_by\_name(1/2) & \\ \hline
H5Olink & \\ \hline
H5Ocopy & \\ \hline
H5Oflush & \\ \hline
H5Orefresh & \\ \hline
H5Oincr\_refcount & \\ \hline
H5Odecr\_refcount & \\ \hline

\end{tabularx}

\end{center}

\subsubsection{H5R interface}

\begin{center}

\textbf{Supported API calls}
\vspace{.2in} \\

\begin{tabularx}{\linewidth}{| X | X |}
\hline
% Center just the header row
\multicolumn{1}{| c |}{\textbf{API call}} & \multicolumn{1}{c |}{\textbf{Notes}} \\ \hline

 & \\ \hline

\end{tabularx}

\textbf{Currently unsupported API calls}
\vspace{.2in} \\

\begin{tabularx}{\linewidth}{| X | X |}
\hline
% Center just the header row
\multicolumn{1}{| c |}{\textbf{API call}} & \multicolumn{1}{c |}{\textbf{Notes}} \\ \hline

H5Rcreate & \\ \hline
H5Rdereference(1/2) & Causes an assertion failure\\ \hline
H5Rget\_name & \\ \hline
H5Rget\_obj\_type(1/2) & \\ \hline
H5Rget\_region & \\ \hline

\end{tabularx}

\end{center}

\newpage

\subsubsection{H5T interface}

\begin{center}

\textbf{Supported API calls}
\vspace{.2in} \\

\begin{tabularx}{\linewidth}{| X | X |}
\hline
% Center just the header row
\multicolumn{1}{| c |}{\textbf{API call}} & \multicolumn{1}{c |}{\textbf{Notes}} \\ \hline

H5Tcommit(1/2) & \\ \hline
H5Tcommit\_anon & \\ \hline
H5Topen(1/2) & \\ \hline
H5Tclose & \\ \hline
H5Tget\_create\_plist & \\ \hline

\end{tabularx}

\textbf{Currently unsupported API calls}
\vspace{.2in} \\

\begin{tabularx}{\linewidth}{| X | X |}
\hline
% Center just the header row
\multicolumn{1}{| c |}{\textbf{API call}} & \multicolumn{1}{c |}{\textbf{Notes}} \\ \hline

H5Tflush & \\ \hline
H5Trefresh & \\ \hline

\end{tabularx}

\end{center}

%\subsection{Feature support}

%TBD

\newpage

\section{Testing}

\subsection{HDF5 and dynamically-loaded VOL connectors}

HDF5 has the capability to dynamically load and use a VOL connector for running tests with. While several HDF5 tests have been updated to take advantage of this capability, please be aware that many of these tests are likely to fail or crash due to their native HDF5-specific nature.

In order to choose a particular VOL connector to use for testing, two initial steps must be taken. First, one must help HDF5 locate the VOL connector by pointing to the directory which contains the built library. This can be accomplished by setting the environment variable \texttt{HDF5\_PLUGIN\_PATH} to this directory. Next, HDF5 needs to know the name of which library to use, which is configured by setting the environment variable \texttt{HDF5\_VOL\_CONNECTOR} to the name of the connector.

In order to use the DAOS VOL connector, the aforementioned environment variables should be set as:

\begin{verbatim}
HDF5_PLUGIN_PATH=/daos/vol/installation/directory/lib
HDF5_VOL_CONNECTOR=daos
\end{verbatim}

Having completed this step, HDF5 will be setup to load the DAOS VOL connector and use it for testing.

\newpage

\subsection{Testing the DAOS VOL connector}

\subsubsection{Generic VOL connector test suite}

In order to test VOL connectors to make sure that they are functioning as expected, a suite of tests which only use the public HDF5 API has been written. This suite of tests is available under the path:

\begin{verbatim}
test/vol/vol_test.c
\end{verbatim}

Currently, this test suite does not have the capability to query what kind of functionality a VOL connector supports and therefore a test will fail if it uses an HDF5 API call which is not implemented, or which is specifically unsupported, in a given VOL connector.

To test the DAOS VOL connector with this test suite, first refer to \hyperref[sec:daos_serv_start]{Starting the DAOS Server} and \hyperref[sec:runtime_env]{Runtime Environment} to make sure that the DAOS Server is up and running and your environment is setup correctly. Once that is done, tests can be run against the DAOS VOL connector
and it should be as simple as running:

\begin{verbatim}
make check
\end{verbatim}

for Autotools builds, or:

\begin{verbatim}
ctest .
\end{verbatim}

for CMake builds\footnote{Automated testing is still under development for CMake builds.}.

\subsubsection{DAOS VOL connector-specific test suite}

There is no DAOS specific test as of this writing.

\newpage
\appendix

\section{Reference Manual}
\label{apdx:ref_manual}
\documentclass[../users_guide.tex]{subfiles}
 
\begin{document}

\section{Reference Manual}
\label{apdx:ref_manual}

\subsection{H5daos\_init}
\label{ref:h5daos_init}

\paragraph{Synopsis:}
\begin{flushleft}%
\begin{minted}[breaklines=true,fontsize=\small]{hdf5-c-lexer.py:HDF5CLexer -x}
herr_t H5daos_init(MPI_Comm pool_comm, uuid_t pool_uuid, const char *pool_grp,
    const char *pool_svcl);
\end{minted}
\end{flushleft}%

\paragraph{Purpose:}
\begin{flushleft}%
Initialize the DAOS VOL connector.
\end{flushleft}%

\paragraph{Description:}
\begin{flushleft}%
\texttt{H5daos\_init} initializes the VOL connector by connecting to the pool
and registering the connector with the library. \texttt{pool\_comm}
identifies the communicator used to connect to the DAOS pool.  This should
include all processes that will participate in I/O. This call is collective
across \texttt{pool\_comm}.
\end{flushleft}%

\paragraph{Parameters:}
\begin{flushleft}%
 \begin{tabular}{ll}%
   \texttt{MPI\_Comm pool\_comm} & IN: Communicator used for connecting to the pool \\
   \texttt{uuid\_t pool\_uuid} & IN: UUID to identify a pool \\
   \texttt{const char *pool\_grp} & IN: Process set name of the DAOS servers managing the pool \\
   \texttt{const char *pool\_svcl} & IN: Comma-separated list of pool service replica ranks \\
 \end{tabular}%
\end{flushleft}%

\paragraph{Returns:}
\begin{flushleft}%
Returns a non-negative value if successful; otherwise returns a negative value.
\end{flushleft}%

%%%%%%%%%%%%%%%%%%%%%%%%%%%%%%%%%%%%%%%%%%%%%%%%%%%%%%%%%%%%%%%%%%%%%%%%%%%%%%
\newpage
\subsection{H5daos\_term}
\label{ref:h5daos_term}

\paragraph{Synopsis:}
\begin{flushleft}%
\begin{minted}[breaklines=true,fontsize=\small]{hdf5-c-lexer.py:HDF5CLexer -x}
herr_t H5daos_term(void);
\end{minted}
\end{flushleft}%

\paragraph{Purpose:}
\begin{flushleft}%
Terminate the DAOS VOL connector.
\end{flushleft}%

\paragraph{Description:}
\begin{flushleft}%
\texttt{H5daos\_term} terminates the DAOS VOL connector.
\end{flushleft}%

\paragraph{Parameters:}
\begin{flushleft}%
None.
\end{flushleft}%

\paragraph{Returns:}
\begin{flushleft}%
Returns a non-negative value if successful; otherwise returns a negative value.
\end{flushleft}%

%%%%%%%%%%%%%%%%%%%%%%%%%%%%%%%%%%%%%%%%%%%%%%%%%%%%%%%%%%%%%%%%%%%%%%%%%%%%%%
\newpage
\subsection{H5Pset\_fapl\_daos}
\label{ref:h5pset_fapl_daos}

\paragraph{Synopsis:}
\begin{flushleft}%
\begin{minted}[breaklines=true,fontsize=\small]{hdf5-c-lexer.py:HDF5CLexer -x}
herr_t H5Pset_fapl_daos(hid_t fapl_id, MPI_Comm comm, MPI_Info info);
\end{minted}
\end{flushleft}%

\paragraph{Purpose:}
\begin{flushleft}%
Set the file access property list to use the DAOS VOL connector.
\end{flushleft}%

\paragraph{Description:}
\begin{flushleft}%
\texttt{H5Pset\_fapl\_daos} modifies the file access property list to use the
DAOS VOL connector. \texttt{file\_comm} and
\texttt{file\_info} identify the communicator and info object used to
coordinate actions on file create, open, flush, and close.
\end{flushleft}%

\paragraph{Parameters:}
\begin{flushleft}%
 \begin{tabular}{ll}%
   \texttt{hid\_t fapl\_id} & IN: File access property list ID \\
   \texttt{MPI\_Comm file\_comm} & IN: MPI Communicator \\
   \texttt{MPI\_Info file\_info} & IN: MPI Info \\
 \end{tabular}%
\end{flushleft}%

\paragraph{Returns:}
\begin{flushleft}%
Returns a non-negative value if successful; otherwise returns a negative value.
\end{flushleft}%

\end{document}


\newpage
\section{Native VOL connector API calls}
\label{apdx:native_calls}

The following HDF5 API calls are specific to the native HDF5 VOL connector and thus are not able to be implemented by the DAOS VOL connector (or other VOL connectors):

\subsection{H5A interface}

\begin{tabularx}{\linewidth}{| X | X |}
\hline
% Center just the header row
\multicolumn{1}{| c |}{\textbf{API call}} & \multicolumn{1}{c |}{\textbf{Notes}} \\ \hline

H5Aiterate1 & Deprecated in favor of H5Aiterate2 \\ \hline
H5Aget\_num\_attrs & Deprecated in favor of H5Oget\_info \\ \hline

\end{tabularx}

\subsection{H5D interface}

\begin{tabularx}{\linewidth}{| X | X |}
\hline
% Center just the header row
\multicolumn{1}{| c |}{\textbf{API call}} & \multicolumn{1}{c |}{\textbf{Notes}} \\ \hline

H5Dformat\_convert & \\ \hline
H5Dget\_chunk\_index\_type & \\ \hline
H5Dget\_chunk\_storage\_size & \\ \hline
H5Dread\_chunk & \\ \hline
H5Dwrite\_chunk & \\ \hline

\end{tabularx}

\subsection{H5F interface}

\begin{tabularx}{\linewidth}{| X | X |}
\hline
% Center just the header row
\multicolumn{1}{| c |}{\textbf{API call}} & \multicolumn{1}{c |}{\textbf{Notes}} \\ \hline

H5Fis\_hdf5 & Uses a default FAPL so can only ever be routed through the native HDF5 VOL connector\\ \hline
H5Fget\_vfd\_handle & \\ \hline
H5Fget\_freespace & \\ \hline
H5Fget\_filesize & \\ \hline
H5Fget\_file\_image & \\ \hline
H5Fget\_mdc\_config & \\ \hline
H5Fset\_mdc\_config & \\ \hline
H5Fget\_mdc\_hit\_rate & \\ \hline
H5Fget\_mdc\_size & \\ \hline
H5Freset\_mdc\_hit\_rate\_stats & \\ \hline
H5Fget\_info(1/2) & \\ \hline
H5Fget\_metadata\_read\_retry\_info & \\ \hline
H5Fget\_free\_sections & \\ \hline
H5Fclear\_elink\_file\_cache & \\ \hline
H5Fstart\_swmr\_write & \\ \hline
H5Fstart\_mdc\_logging & \\ \hline
H5Fstop\_mdc\_logging & \\ \hline
H5Fget\_mdc\_logging\_status & \\ \hline
H5Fset\_libver\_bounds & \\ \hline
H5Fformat\_convert & \\ \hline
H5Freset\_page\_buffering\_stats & \\ \hline
H5Fget\_page\_buffering\_stats & \\ \hline
H5Fget\_mdc\_image\_info & \\ \hline
H5Fget\_eoa & \\ \hline
H5Fincrement\_filesize & \\ \hline
H5Fget\_dset\_no\_attrs\_hint & \\ \hline
H5Fset\_dset\_no\_attrs\_hint & \\ \hline
H5Fset\_latest\_format & \\ \hline
H5Fset\_mpi\_atomicity & \\ \hline
H5Fget\_mpi\_atomicity & \\ \hline

\end{tabularx}

\subsection{H5G interface}

\begin{tabularx}{\linewidth}{| X | X |}
\hline
% Center just the header row
\multicolumn{1}{| c |}{\textbf{API call}} & \multicolumn{1}{c |}{\textbf{Notes}} \\ \hline

H5Gset\_comment & Deprecated in favor of H5Oset\_comment/H5Oset\_comment\_by\_name \\ \hline
H5Gget\_comment & Deprecated in favor of H5Oget\_comment/H5Oget\_comment\_by\_name \\ \hline
H5Giterate & Deprecated in favor of H5Literate \\ \hline
H5Gget\_objinfo & Deprecated in favor of H5Lget\_info/H5Oget\_info \\ \hline
H5Gget\_objtype\_by\_idx & Deprecated in favor of H5Lget\_info/H5Oget\_info \\ \hline

\end{tabularx}

\subsection{H5L interface}

\begin{tabularx}{\linewidth}{| X | X |}
\hline
% Center just the header row
\multicolumn{1}{| c |}{\textbf{API call}} & \multicolumn{1}{c |}{\textbf{Notes}} \\ \hline

& \\ \hline

\end{tabularx}

\subsection{H5O interface}

\begin{tabularx}{\linewidth}{| X | X |}
\hline
% Center just the header row
\multicolumn{1}{| c |}{\textbf{API call}} & \multicolumn{1}{c |}{\textbf{Notes}} \\ \hline

H5Oget\_info(1/2) & \\ \hline
H5Oget\_info\_by\_name(1/2) & \\ \hline
H5Oget\_info\_by\_idx(1/2) & \\ \hline
H5Oset\_comment(\_by\_name) & Deprecated in favor of using attributes on objects \\ \hline
H5Oget\_comment(\_by\_name) & Deprecated in favor of using attributes on objects \\ \hline
H5Oare\_mdc\_flushes\_disabled & \\ \hline
H5Oenable\_mdc\_flushes & \\ \hline
H5Odisable\_mdc\_flushes & \\ \hline

\end{tabularx}

\subsection{H5R interface}

\begin{tabularx}{\linewidth}{| X | X |}
\hline
% Center just the header row
\multicolumn{1}{| c |}{\textbf{API call}} & \multicolumn{1}{c |}{\textbf{Notes}} \\ \hline

& \\ \hline

\end{tabularx}

\subsection{H5T interface}

\begin{tabularx}{\linewidth}{| X | X |}
\hline
% Center just the header row
\multicolumn{1}{| c |}{\textbf{API call}} & \multicolumn{1}{c |}{\textbf{Notes}} \\ \hline

& \\ \hline

\end{tabularx}

\newpage
\section*{Revision History}

\begin{tabularx}{\linewidth}{| l | X |}
\hline
March 29, 2019 & First Draft \\
\hline
& \\
\hline
& \\
\hline
& \\
\hline
& \\
\hline
& \\
\hline
& \\
\hline
& \\
\hline
& \\
\hline
\end{tabularx}

\end{document}
