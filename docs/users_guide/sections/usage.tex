\documentclass[../users_guide.tex]{subfiles}
 
\begin{document}

\section{Using the DAOS VOL connector within an HDF5 application}

This section outlines the unique aspects of writing, building and running
\acrshort{hdf5} applications with the \dvc{}.

\subsection{Building the HDF5 DAOS VOL connector}

The following is a quick set of instructions for building the HDF5 DAOS VOL connector. Note that these instructions are not comprehensive and may be slightly out of date; please refer to the HDF5 DAOS VOL connector's \href{https://bitbucket.hdfgroup.org/projects/HDF5VOL/repos/daos-vol/browse/README.md}{README} file for the most updated building instructions.

The HDF5 DAOS VOL connector is built using CMake. CMake version 2.8.12.2 or greater is required to build the connector itself, but version 3.1 or greater is required to build the connector's tests. To build the connector, first one should create a build directory within the source tree:

\begin{verbatim}
cd daos-vol
mkdir build
cd build
\end{verbatim}

After that, if all of the required components (DAOS, CaRT, MPI and HDF5) are located within the system path, building the connector should be as simple as running the following two commands to have CMake generate the build files for \texttt{make} to use.

\begin{verbatim}
cmake ..
make && make install
\end{verbatim}

Some notable CMake variables are listed below. These can be used to control the build process and can be supplied to the \texttt{cmake} command by prepending them with \texttt{-D}. Some of the connector-specific options may be needed if the required components mentioned previously cannot be found within the system path.

CMake-specific options:

\begin{itemize}
  \item \texttt{CMAKE\_INSTALL\_PREFIX} - This variable controls the install directory that the resulting output files are written to.
  \item \texttt{CMAKE\_BUILD\_TYPE} - This variable controls the type of build used for the VOL connector. Valid values are Release, Debug, RelWithDebInfo and MinSizeRel.
\end{itemize}

DAOS VOL connector-specific options:

\begin{itemize}
  \item \texttt{BUILD\_TESTING} - This variable is used to enable/disable building of the DAOS VOL connector's tests.
  \item \texttt{BUILD\_EXAMPLES} - This variable is used to enable/disable building of the DAOS VOL connector's HDF5 examples.
  \item \texttt{CART\_INCLUDE\_DIR} - This variable controls the CaRT include directory used by the VOL connector build process. Used in conjunction with the \texttt{CART\_LIBRARY} variable.
  \item \texttt{CART\_LIBRARY} - This variable controls the CaRT library used by the VOL connector build process. It should be set to the full path to the CaRT library, including the library's name (e.g., /path/libcart.so). Used in conjunction with the \texttt{CART\_INCLUDE\_DIR} variable.
  \item \texttt{DAOS\_LIBRARY} - This variable controls the DAOS library used by the VOL connector build process. It should be set to the full path to the DAOS library, including the library's name (e.g., /path/libdaos.so). Used in conjunction with the \texttt{DAOS\_COMMON\_LIBRARY} and \texttt{DAOS\_INCLUDE\_DIR} variables.
  \item \texttt{DAOS\_COMMON\_LIBRARY} - This variable controls the DAOS 'common' library used by the VOL connector build process. It should be set to the full path to the DAOS common library, including the library's name (e.g., /path/libdaos\_common.so). Used in conjunction with the \texttt{DAOS\_LIBRARY} and \texttt{DAOS\_INCLUDE\_DIR} variables.
  \item \texttt{DAOS\_INCLUDE\_DIR} - This variable controls the DAOS include directory used by the VOL connector build process. Used in conjunction with the \texttt{DAOS\_LIBRARY} and \texttt{DAOS\_COMMON\_LIBRARY} variables.
  \item \texttt{MPI\_C\_COMPILER} - This variable controls the MPI C Compiler used by the VOL connector build process. It should be set to the full path to the MPI C Compiler, including the name of the executable.
  \item \texttt{HDF5\_C\_COMPILER\_EXECUTABLE} - This variable controls the HDF5 compiler wrapper script used by the VOL connector build process. It should be set to the full path to the HDF5 compiler wrapper, including the name of the wrapper script. The following two variables may also need to be set.
  \item \texttt{HDF5\_C\_LIBRARY\_hdf5} - This variable controls the HDF5 library used by the VOL connector build process. It should be set to the full path to the HDF5 library, including the library's name (e.g., /path/libhdf5.so). Used in conjunction with the \texttt{HDF5\_C\_INCLUDE\_DIR} variable.
  \item \texttt{HDF5\_C\_INCLUDE\_DIR} - This variable controls the HDF5 include directory used by the VOL connector build process. Used in conjunction with the \texttt{HDF5\_C\_LIBRARY\_hdf5} variable.
\end{itemize}

\subsection{Writing HDF5 DAOS VOL connector applications}

There are currently two main ways to tell an existing \acrshort{hdf5} application to use
the \dvc{}: either \textit{implicitly} by using environment
variables to tell the \acrshort{hdf5} library to load the connector as a dynamically loaded
plugin or \textit{explicitly} by making use of \acrshort{hdf5} property lists.

\subsubsection{With the DAOS VOL connector as a dynamically-loaded plugin}
\label{sec:dynamic_plugin}

\acrshort{hdf5} has the capability to dynamically load and use a \vc{} for running
applications with. In order to choose a particular \vc{} to use, two
initial steps must be taken. First, one must help \acrshort{hdf5} locate the \vc{}
by pointing to the directory which contains the built library. This can be
accomplished by setting the environment variable \texttt{HDF5\_PLUGIN\_PATH} to
this directory. Next, \acrshort{hdf5} needs to know the name of which library to use, which
is configured by setting the environment variable \texttt{HDF5\_VOL\_CONNECTOR}
to the name of the connector.

In order to use the \dvc{}, the aforementioned environment variables 
should be set as:

\begin{verbatim}
HDF5_PLUGIN_PATH=/daos/vol/installation/directory/lib
HDF5_VOL_CONNECTOR=daos
\end{verbatim}

Having completed this step, \acrshort{hdf5} will be setup to load the \dvc{}
and use it for running applications, including \acrshort{hdf5}'s own tests.
No additional modifications will need to be made to the existing \acrshort{hdf5} application.

\subsubsection{Without the DAOS VOL connector as a dynamically-loaded plugin}

If dynamic loading of the \dvc{} is not used, any \acrshort{hdf5} application
using the connector must:
\begin{enumerate}
 \item Include \texttt{daos\_vol\_public.h}, found in the \texttt{include}
directory of the \dvc{} installation directory.
 \item Link against \texttt{libhdf5\_vol\_daos.so} (or similar), found in
the \texttt{lib} directory of the \dvc{} installation directory, and
against \texttt{libuuid.so} (or similar) in order to use UUIDs. Note that dependencies
can alternatively be retrieved through CMake or pkg-config.
\end{enumerate}

An \acrshort{hdf5} \dvc{} application also requires in that particular case three new
function calls in addition to those for an equivalent \acrshort{hdf5} application (see
Appendix~\ref{apdx:ref_manual} for more details):

\begin{itemize}
 \item \texttt{\hyperref[ref:h5daos_init]{H5daos\_init()}} --- Initializes the \dvc{}

    Called upon application startup, before any file is accessed.

 \item \texttt{\hyperref[ref:h5pset_fapl_daos]{H5Pset\_fapl\_daos()}} --- Set \dvc{} access on File Access Property List.

    Called to prepare a FAPL to open a file through the \dvc{}. See \href{https://support.hdfgroup.org/HDF5/Tutor/property.html#fa}{HDF5 File Access Property Lists} for more information about File Access Property Lists.

 \item \texttt{\hyperref[ref:h5daos_term]{H5daos\_term()}} --- Cleanly shutdown the \dvc{}

    Called on application shutdown, after all files have been closed.
\end{itemize}

\subsubsection{Skeleton Example}

Below is a no-op application that opens and closes a file using the \dvc{}.
For clarity, no error-checking is performed. Note that this example is
meant only for the case when the \dvc{} is not being dynamically loaded.

\begin{minted}[breaklines=true,fontsize=\small]{hdf5-c-lexer.py:HDF5CLexer -x}
#include "hdf5.h"
#include "daos_vol_public.h"

int main(void)
{
    uuid_t pool_uuid;
    hid_t fapl_id, file_id;

    /* Parse the pool UUID. */
    uuid_parse("fce30f79-b34b-46c1-9b1f-bb52d99dacca", pool_uuid);

    /* Initialize DAOS VOL connector using MPI_COMM_WORLD for the pool
     * communicator, the above parsed UUID for the pool UUID, "daos_server"
     * as the group name for the DAOS servers managing the pool and
     * simply rank 0 as the only rank in the pool service list. */
    H5daos_init(MPI_COMM_WORLD, pool_uuid, "daos_server", "0");

    fapl_id = H5Pcreate(H5P_FILE_ACCESS);
    H5Pset_fapl_daos(fapl_id, MPI_COMM_WORLD, MPI_INFO_NULL);

    /* Currently required for the DAOS VOL connector, set all metadata
     * operations to be collective */
    H5Pset_all_coll_metadata_ops(fapl_id, true); 

    file_id = H5Fopen("my_file.h5", H5F_ACC_RDWR, fapl_id);

    /* Operate on file */
    [...]

    H5Pclose(fapl_id);
    H5Fclose(file_id);

    /* Terminate the DAOS VOL connector. */
    H5daos_term();

    return 0;
}
\end{minted}

\subsection{Building HDF5 DAOS VOL connector applications}

Assuming an \acrshort{hdf5} application has been written following the instructions in the previous section, the application must be built prior to running. In general, the
application should be built as normal for any other \acrshort{hdf5} application. However,
if the \dvc{} is not being dynamically loaded, the steps in the following section are required to build the application.

\subsubsection{Without the DAOS VOL connector as a dynamically-loaded plugin}

To link in the required libraries, the compiler will likely require the
additional linker flags:

\begin{verbatim}
-lhdf5_vol_daos -luuid
\end{verbatim}

However, these flags may vary depending on platform, compiler and installation
location of the \dvc{}. It is highly recommended that compilation
of \acrshort{hdf5} \dvc{} applications be done using either the
\texttt{h5cc/h5pcc} script included with \acrshort{hdf5} distributions, or CMake,
pkg-config, as these will manage linking with the \acrshort{hdf5} library.

If \acrshort{hdf5} was built using autotools, this script will be called \texttt{h5pcc} and
may be found in the \texttt{bin} directory of the \acrshort{hdf5} installation. If \acrshort{hdf5}
was built with CMake, this script will simply be called \texttt{h5cc} and can
be found in the same location. The above notice about additional library
linking applies to usage of \texttt{h5cc/h5pcc}. For example:
\begin{verbatim}
h5cc/h5pcc -lhdf5_vol_daos -luuid my_application.c -o my_application
\end{verbatim}

\subsection{Running HDF5 DAOS VOL connector applications}
\label{running_daos_vol_apps}

Running applications that use the \dvc{} connector requires access to a \acrshort{daos}
server. Refer to
\href{https://daos-stack.github.io/admin/installation/}{DAOS Software Installation}
for more information on the setup process for this. For the \dvc{}
to correctly interact with a \acrshort{daos} server instance, the server must be \hyperref[sec:daos_serv_start]{running} and it must be passed the UUID of the
\acrshort{daos} pool to use and a list of \acrshort{daos} pool service list ranks, as detailed in the
following sections.

\subsubsection{Starting the DAOS Server}
\label{sec:daos_serv_start}

Instructions for starting a \acrshort{daos} Server can be found in the \href{https://daos-stack.github.io/admin/deployment/#server-startup}{DAOS Documentation}.

\subsubsection{With the DAOS VOL connector as a dynamically-loaded plugin}

If the \dvc{} is dynamically loaded by \acrshort{hdf5}, the \acrshort{daos} pool UUID and
\acrshort{daos} pool service rank list are passed via the two environment variables below.

\begin{verbatim}
DAOS_POOL - The UUID of the DAOS pool to use.

DAOS_SVCL - A comma-separated list of server ranks used for
            daos_pool_connect(). Generated from daos_pool_create().
\end{verbatim}

\subsubsection{Without the connector as a dynamically-loaded plugin}

If the \dvc{} is not being dynamically loaded, the \acrshort{daos} pool UUID
and \acrshort{daos} pool service rank list should be passed via the call to
\hyperref[ref:h5daos_init]{H5daos\_init()} within the application.

\subsubsection{Example Applications}

Some of the example C applications which are included with \acrshort{hdf5} distributions have been adapted to work with the \dvc{} and are included under the top-level \href{https://bitbucket.hdfgroup.org/projects/HDF5VOL/repos/daos-vol/browse/examples}{\texttt{examples}} directory in the \dvc{} source root directory. The built example applications can be run from the \texttt{bin} directory inside the build directory.

In addition to these examples, the \href{https://bitbucket.hdfgroup.org/projects/HDF5VOL/
repos/daos-vol/browse/test}{\texttt{test/vol}} directory contains several test
files, each containing test functions that are examples of \acrshort{hdf5} applications in
miniature, focused on a particular behavior. These mini-application tests cover a moderate 
amount of \acrshort{hdf5}'s public API functionality and should be a good indicator of
whether the \dvc{} is working correctly in conjunction with a
running \acrshort{daos} API-aware instance. Note that these tests currently rely on \acrshort{hdf5}'s
dynamically-loaded \vc{} capabilities in order to run with the \dvc{}.

\end{document}
