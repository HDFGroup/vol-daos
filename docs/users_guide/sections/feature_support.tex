\documentclass[../users_guide.tex]{subfiles}
 
\begin{document}

\section{HDF5 API Support}

\subsection{Feature Specific Support}

The following sections serve to illustrate the \dvc{}'s support for features in \acrshort{hdf5}, as well as to highlight any differences between the expected behavior of an \acrshort{hdf5} feature versus the actual behavior as implemented by the VOL connector.

\subsubsection{Attribute Features}

\begin{tabularx}{\linewidth}{| X | X | X | X | >{\RaggedRight}X |}
\hline
\rowcolor{lightgray!50}%
% Center just the header row
\multicolumn{3}{| c |}{\textbf{Feature}} & \multicolumn{1}{c |}{\textbf{Supported?}} & \multicolumn{1}{c |}{\textbf{Notes}} \\ \hline

\multirow[c]{3}{\linewidth}{Dataspace} & \multirow[c]{3}{\linewidth}{Dimensionality} & H5S\_NULL & Yes & \\ \cline{3-4}
& & H5S\_SCALAR & Yes & \\ \cline{3-4}
& & SIMPLE & Yes & \\ \cline{3-4} \hline

\multicolumn{2}{| l |}{\multirow[c]{6}{*}[-90pt]{Datatype}} & Atomic & Yes &
    \multirow[c]{5}{\linewidth}{Variable-length of array datatypes and variable-length datatypes inside compound datatypes are currently unsupported. Datatype conversion for the parent datatype is currently unsupported.\footnotemark[1]} \\ [30pt] \cline{3-4}
\multicolumn{2}{| l |}{} & Compound & Yes & \\ [30pt] \cline{3-4}
\multicolumn{2}{| l |}{} & Variable-length & Yes & \\ [30pt] \cline{3-4}
\multicolumn{2}{| l |}{} & Array & Yes & \\ [30pt] \cline{3-4}
\multicolumn{2}{| l |}{} & Opaque & Yes & \\ [30pt] \cline{3-4}
\multicolumn{2}{| l |}{} & Reference & No\footnotemark[1] & \\ [30pt] \cline{3-4}
\hline

\multirow[c]{2}{\linewidth}{Properties} & \multirow[c]{2}{\linewidth}{Name Encoding} & ASCII & Yes & \\ \cline{3-5}
& & UTF-8 & No & \\ \cline{3-4} \hline

\end{tabularx}

\footnotetext[1]{Will be supported by end of Q4 2019.}

\newpage

\subsubsection{Dataset Features}

\begin{tabularx}{\linewidth}{| X | X | X | X | >{\RaggedRight}X |}
\hline
\rowcolor{lightgray!50}%
% Center just the header row
\multicolumn{3}{| c |}{\textbf{Feature}} & \multicolumn{1}{c |}{\textbf{Supported?}} & \multicolumn{1}{c |}{\textbf{Notes}} \\ \hline

\multirow[c]{7}{\linewidth}[-45pt]{Dataspace} & \multirow[c]{3}{\linewidth}{Dimensionality} & H5S\_NULL & Yes & \\ \cline{3-4}
& & H5S\_SCALAR & Yes & \\ \cline{3-4}
& & SIMPLE & Yes & \\ \cline{3-4} \cline{2-5}
& \multirow[c]{4}{\linewidth}[-55pt]{Selection Type} & NONE & Yes & \multirow[c]{4}{\linewidth}{For chunked datasets, the selection currently must be the same shape in the memory and file dataspaces.\footnotemark[1]} \\ [28pt] \cline{3-4}
& & H5S\_ALL & Yes & \\ [28pt] \cline{3-4}
& & Hyperslab Selection & Yes & \\ [28pt] \cline{3-4}
& & Point Selection & Yes & \\ [28pt] \cline{3-4} \hline

\multicolumn{2}{| l |}{\multirow[c]{6}{*}[-90pt]{Datatype}} & Atomic & Yes & \multirow[c]{5}{\linewidth}{Variable-length of array datatypes and variable-length datatypes inside compound datatypes are currently unsupported. Datatype conversion for the parent datatype is currently unsupported.\footnotemark[1]} \\ [30pt] \cline{3-4}
\multicolumn{2}{| l |}{} & Compound & Yes & \\ [30pt] \cline{3-4}
\multicolumn{2}{| l |}{} & Variable-length & Yes & \\ [30pt] \cline{3-4}
\multicolumn{2}{| l |}{} & Array & Yes & \\ [30pt] \cline{3-4}
\multicolumn{2}{| l |}{} & Opaque & Yes & \\ [30pt] \cline{3-4}
\multicolumn{2}{| l |}{} & Reference & No\footnotemark[1] & \\ [30pt] \cline{3-4}
\hline

\end{tabularx}

\footnotetext[1]{Will be supported by end of Q4 2019.}

\newpage

\begin{tabularx}{\linewidth}{| X | X | X | X | >{\RaggedRight}X |}
\hline
\rowcolor{lightgray!50}%
% Center just the header row
\multicolumn{3}{| c |}{\textbf{Feature}} & \multicolumn{1}{c |}{\textbf{Supported?}} & \multicolumn{1}{c |}{\textbf{Notes}} \\ \hline

\multirow[c]{9}{\linewidth}[-80pt]{Properties} & \multirow[c]{5}{\linewidth}[-35pt]{Storage Properties (creation)} & Compact & No & Setting is ignored; stored as contiguous. \\ \cline{3-5}
& & External & No & Setting is ignored; stored as contiguous. \\ \cline{3-5}
& & Contiguous & Yes & Default storage type. \\ \cline{3-5}
& & Chunked & Yes & \\ \cline{3-5}
& & VDS & No & \\ \cline{2-5}
& \multirow[c]{4}{\linewidth}[-45pt]{Other Properties (creation)} & Attribute Creation Order & Yes & In order to work correctly, the attribute creation order feature requires that a file is touched collectively. \\ \cline{3-5}
& & Fill Value & No & \\ \cline{3-5}
& & Filters & No & May be supported in the future. \\ \cline{3-5}
& & Storage Allocation Time & N/A & \\ \hline

\end{tabularx}

\newpage

\begin{tabularx}{\linewidth}{| X | X | X | X | >{\RaggedRight}X |}
\hline
\rowcolor{lightgray!50}%
% Center just the header row
\multicolumn{3}{| c |}{\textbf{Feature}} & \multicolumn{1}{c |}{\textbf{Supported?}} & \multicolumn{1}{c |}{\textbf{Notes}} \\ \hline

\multirow[c]{4}{\linewidth}[-55pt]{Properties (cont.)} & \multirow[c]{3}{\linewidth}[-50pt]{Access Properties} & Chunk cache & No & May be supported in the future. \\ \cline{3-5}
& & VDS views and printf & No & \\ \cline{3-5}
& & MPI-I/O Collective Metadata Ops & Yes & By default, all metadata operations are collective for writes and independent for reads.\footnotemark[2] \\ \cline{2-5}
& \multirow[c]{1}{\linewidth}[-10pt]{Transfer Properties} & MPI-I/O Independent or Collective I/O mode & N/A & \\ \hline

\end{tabularx}

\footnotetext[2]{Independent metadata writes will be supported in the future.}

\subsubsection{File Features}

\begin{tabularx}{\linewidth}{| X | X | X | X | >{\RaggedRight}X |}
\hline
\rowcolor{lightgray!50}%
% Center just the header row
\multicolumn{3}{| c |}{\textbf{Feature}} & \multicolumn{1}{c |}{\textbf{Supported?}} & \multicolumn{1}{c |}{\textbf{Notes}} \\ \hline

\multicolumn{2}{| l |}{\multirow[c]{2}{*}{File creation flags}} & \multicolumn{1}{l |}{H5F\_ACC\_TRUNC} & Yes & \multirow[c]{4}{\linewidth}{The file creation flags behave as for native HDF5.} \\ \cline{3-4}
\multicolumn{2}{| l |}{} & \multicolumn{1}{l |}{H5F\_ACC\_EXCL} & Yes & \\ \cline{1-4}
\multicolumn{2}{| l |}{\multirow[c]{2}{*}{File opening flags}} & \multicolumn{1}{l |}{H5F\_ACC\_RDWR} & Yes & \\ \cline{3-4}
\multicolumn{2}{| l |}{} & \multicolumn{1}{l |}{H5F\_ACC\_RDONLY} & Yes & \\ \hline

\multirow[c]{8}{\linewidth}[-100pt]{Properties} & \multirow[c]{1}{\linewidth}[-70pt]{Creation Properties} & Attribute Creation Order & Yes & In order to work correctly, the attribute creation order feature requires that a file is touched collectively. The rest of the file creation properties are related to the native HDF5-specific file format. \\ \cline{2-5}

& \multirow[c]{7}{\linewidth}{Access Properties (Drivers)} & SEC2 Driver & N/A & \multirow[c]{4}{\linewidth}{These drivers are applicable to native HDF5 only.} \\ \cline{3-4}
& & Family Driver & N/A & \\ \cline{3-4}
& & Split Driver & N/A & \\ \cline{3-4}
& & Multi Driver & N/A & \\ \cline{3-4}
& & Core Driver & N/A & \\ \hline

\end{tabularx}

\newpage

\begin{tabularx}{\linewidth}{| X | X | X | X | >{\RaggedRight}X |}
\hline
\rowcolor{lightgray!50}%
% Center just the header row
\multicolumn{3}{| c |}{\textbf{Feature}} & \multicolumn{1}{c |}{\textbf{Supported?}} & \multicolumn{1}{c |}{\textbf{Notes}} \\ \hline

\multirow[c]{10}{\linewidth}[-100pt]{Properties (cont.)} & \multirow[c]{2}{\linewidth}[-35pt]{Access Properties (Drivers, cont.)} & Log Driver & N/A & \\ \cline{3-5}
& & MPI-I/O & Yes & This property just indicates parallel access to the file; it doesn't use HDF5 MPI I/O driver underneath. \\ \cline{2-5}

& \multirow[c]{8}{\linewidth}[-60pt]{Access Properties (Other)} & MPI-I/O Collective Metadata Ops & Yes & By default, all metadata operations are collective for writes and independent for reads.\footnotemark[2] \\ \cline{3-5}
& & User block & N/A & \\ \cline{3-5}
& & Chunk Cache & No & May be supported in the future. \\ \cline{3-5}
& & Object flushing callbacks & N/A & \\ \cline{3-5}
& & File closing degree & N/A & \\ \cline{3-5}
& & Evict on close & N/A & \\ \cline{3-5}
& & Sieve buffer size for partial I/O & No & May be supported in the future. \\ \cline{3-5}
& & File Image & N/A & \\ \hline

\end{tabularx}

\footnotetext[2]{Independent metadata writes will be supported in the future.}

\subsubsection{Group Features}

\begin{tabularx}{\linewidth}{| X | X | >{\RaggedRight}X | X | >{\RaggedRight}X |}
\hline
\rowcolor{lightgray!50}%
% Center just the header row
\multicolumn{3}{| c |}{\textbf{Feature}} & \multicolumn{1}{c |}{\textbf{Supported?}} & \multicolumn{1}{c |}{\textbf{Notes}} \\ \hline

\multirow[c]{4}{\linewidth}[-120pt]{Properties} & \multirow[c]{3}{\linewidth}[-100pt]{Creation Properties} & Link Creation Order & Yes & In order to work correctly, the link creation order feature requires that a file is touched collectively. \\ \cline{3-5}
& & Attribute Creation Order & Yes & In order to work correctly, the attribute creation order feature requires that a file is touched collectively. \\ \cline{3-5}
& & Other Properties & N/A & These properties are related to the native HDF5-specific file format. \\ \cline{2-5}
& \multirow[c]{1}{\linewidth}[-35pt]{Access Properties} & MPI-I/O Collective Metadata Ops & Yes & By default, all metadata operations are collective for writes and independent for reads.\footnotemark[2] \\ \hline

\end{tabularx}

\footnotetext[2]{Independent metadata writes will be supported in the future.}

\newpage

\subsection{API Specific Support}

The following sections serve to illustrate the \dvc{}'s support for the \acrshort{hdf5} API, as well as to highlight any differences between the expected behavior of an \acrshort{hdf5} API call versus the actual behavior as implemented by the VOL connector. If a particular \acrshort{hdf5} API call does not appear among these tables, it is most likely a native \acrshort{hdf5}-specific API call which cannot be implemented by non-native \acrshort{hdf5} VOL connectors. These types of API calls are listed among the tables in Appendix~\ref{apdx:native_calls}.

\newpage

\subsubsection{H5A interface}

\begin{center}

\textbf{Supported API calls}
\vspace{.2in} \\

\begin{tabularx}{\linewidth}{| X | >{\RaggedRight}X |}
\hline
\rowcolor{lightgray!50}%
% Center just the header row
\multicolumn{1}{| c |}{\textbf{API call}} & \multicolumn{1}{c |}{\textbf{Notes}} \\ \hline

H5Acreate(1/2) & \\ \hline
H5Acreate\_by\_name & \\ \hline
H5Aopen(\_by\_name/\_by\_idx) & For H5Aopen\_by\_idx, \texttt{H5\_ITER\_DEC} is currently unsupported for the index ordering when \texttt{H5\_INDEX\_NAME} is used for the index type\\ \hline
H5Aopen\_idx & Deprecated in favor of H5A\_open\_by\_idx\\ \hline
H5Aopen\_name & Deprecated in favor of H5A\_open\_by\_name\\ \hline
H5Awrite & \\ \hline
H5Aread & \\ \hline
H5Aclose & \\ \hline
H5Aiterate(2) & \begin{itemize}
                    \item Restarting iteration from an index value is currently unsupported
                    \item \texttt{H5\_ITER\_DEC} is currently unsupported for the index ordering when \texttt{H5\_INDEX\_NAME} is used for the index type
                \end{itemize}\\ \hline
H5Aiterate\_by\_name & \begin{itemize}
                           \item Restarting iteration from an index value is currently unsupported
                           \item \texttt{H5\_ITER\_DEC} is currently unsupported for the index ordering when \texttt{H5\_INDEX\_NAME} is used for the index type
                       \end{itemize}\\ \hline
H5Aexists(\_by\_name) & \\ \hline
H5Arename(\_by\_name) & \\ \hline
H5Adelete(\_by\_name/\_by\_idx) & For H5Adelete\_by\_idx, \texttt{H5\_ITER\_DEC} is currently unsupported for the index ordering when \texttt{H5\_INDEX\_NAME} is used for the index type\\ \hline
H5Aget\_name(\_by\_idx) & For H5Aget\_name\_by\_idx, \texttt{H5\_ITER\_DEC} is currently unsupported for the index ordering when \texttt{H5\_INDEX\_NAME} is used for the index type\\ \hline
\end{tabularx}

\begin{tabularx}{\linewidth}{| X | >{\RaggedRight}X |}
\hline
\rowcolor{lightgray!50}%
% Center just the header row
\multicolumn{1}{| c |}{\textbf{API call}} & \multicolumn{1}{c |}{\textbf{Notes}} \\ \hline

H5Aget\_space & \\ \hline
H5Aget\_type & \\ \hline
H5Aget\_info(\_by\_name/\_by\_idx) & Of the four fields in the \texttt{H5A\_info\_t} struct:
                                     \begin{itemize}
                                         \item \texttt{corder\_valid} is set to TRUE only if attribute creation order tracking is enabled for the object containing the attribute; it is set to FALSE otherwise
                                         \item \texttt{corder} is set appropriately if attribute creation order tracking is enabled for the object containing the attribute; it is set to 0 otherwise
                                         \item \texttt{cset} is currently always set to \texttt{H5T\_CSET\_ASCII}
                                         \item \texttt{data\_size} is set appropriately
                                     \end{itemize}

                                     For H5Aget\_info\_by\_idx, \texttt{H5\_ITER\_DEC} is currently unsupported for the index ordering when \texttt{H5\_INDEX\_NAME} is used for the index type\\ \hline
H5Aget\_create\_plist & \\ \hline

\end{tabularx}

\textbf{Currently unsupported API calls}
\vspace{.1in} \\

\begin{tabularx}{\linewidth}{| X | >{\RaggedRight}X |}
\hline
\rowcolor{lightgray!50}%
% Center just the header row
\multicolumn{1}{| c |}{\textbf{API call}} & \multicolumn{1}{c |}{\textbf{Notes}} \\ \hline

H5Aget\_storage\_size & \\ \hline

\end{tabularx}

\end{center}

\newpage

\subsubsection{H5D interface}

\begin{center}

\textbf{Supported API calls}
\vspace{.2in} \\

\begin{tabularx}{\linewidth}{| X | >{\RaggedRight}X |}
\hline
\rowcolor{lightgray!50}%
% Center just the header row
\multicolumn{1}{| c |}{\textbf{API call}} & \multicolumn{1}{c |}{\textbf{Notes}} \\ \hline

H5Dcreate(1/2) & \\ \hline
H5Dcreate\_anon & \\ \hline
H5Dopen(1/2) & \\ \hline
H5Dwrite & \\ \hline
H5Dread & \\ \hline
H5Dclose & \\ \hline
H5Dextend & Upon dataset expansion or shrinking, data is currently left uninitialized instead of being initialized to 0\\ \hline
H5Dset\_extent & Upon dataset expansion or shrinking, data is currently left uninitialized instead of being initialized to 0\\ \hline
H5Dget\_space & \\ \hline
H5Dget\_type & \\ \hline
H5Dget\_create\_plist & \\ \hline
H5Dget\_access\_plist & \\ \hline

\end{tabularx}

\textbf{Currently unsupported API calls}
\vspace{.2in} \\

\begin{tabularx}{\linewidth}{| X | >{\RaggedRight}X |}
\hline
\rowcolor{lightgray!50}%
% Center just the header row
\multicolumn{1}{| c |}{\textbf{API call}} & \multicolumn{1}{c |}{\textbf{Notes}} \\ \hline

H5Dflush & \\ \hline
H5Drefresh & \\ \hline
H5Dget\_storage\_size & \\ \hline
H5Dget\_space\_status & Space status is currently always set to \texttt{H5D\_SPACE\_STATUS\_NOT\_ALLOCATED}\\ \hline
H5Dget\_offset & \\ \hline

\end{tabularx}

\end{center}

\newpage

\subsubsection{H5F interface}

\begin{center}

\textbf{Supported API calls}
\vspace{.2in} \\

\begin{tabularx}{\linewidth}{| X | >{\RaggedRight}X |}
\hline
\rowcolor{lightgray!50}%
% Center just the header row
\multicolumn{1}{| c |}{\textbf{API call}} & \multicolumn{1}{c |}{\textbf{Notes}} \\ \hline

H5Fcreate & \\ \hline
H5Fopen & \\ \hline
H5Freopen & \\ \hline
H5Fis\_accessible & \\ \hline
H5Fget\_create\_plist & \\ \hline
H5Fget\_access\_plist & \\ \hline
H5Fget\_intent & \\ \hline
H5Fget\_name & \\ \hline
H5Fget\_obj\_count & \\ \hline
H5Fget\_obj\_ids & \\ \hline
H5Fclose & \\ \hline

\end{tabularx}

\textbf{Currently unsupported API calls}
\vspace{.2in} \\

\begin{tabularx}{\linewidth}{| X | >{\RaggedRight}X |}
\hline
\rowcolor{lightgray!50}%
% Center just the header row
\multicolumn{1}{| c |}{\textbf{API call}} & \multicolumn{1}{c |}{\textbf{Notes}} \\ \hline

H5Fflush & \\ \hline
H5Fmount & \\ \hline
H5Funmount & \\ \hline

\end{tabularx}

\end{center}

\newpage

\subsubsection{H5G interface}

\begin{center}

\textbf{Supported API calls}
\vspace{.2in} \\

\begin{tabularx}{\linewidth}{| X | >{\RaggedRight}X |}
\hline
\rowcolor{lightgray!50}%
% Center just the header row
\multicolumn{1}{| c |}{\textbf{API call}} & \multicolumn{1}{c |}{\textbf{Notes}} \\ \hline

H5Gcreate(1/2) & \\ \hline
H5Gcreate\_anon & \\ \hline
H5Gopen(1/2) & \\ \hline
H5Gclose & \\ \hline
H5Gunlink & \\ \hline
H5Gget\_create\_plist & \\ \hline
H5Gget\_info(\_by\_name/\_by\_idx) & Of the four fields in the \texttt{H5G\_info\_t} struct:
                                     \begin{itemize}
                                         \item \texttt{storage\_type} is always set to \texttt{H5G\_STORAGE\_TYPE\_UNKNOWN}
                                         \item \texttt{nlinks} is set appropriately
                                         \item \texttt{max\_corder} is set appropriately if link creation order is tracked for the group
                                         \item \texttt{mounted} is currently always set to \texttt{FALSE}
                                     \end{itemize}
                                     For \texttt{H5Gget\_info\_by\_idx}, \texttt{H5\_ITER\_DEC} is currently unsupported for the index ordering when \texttt{H5\_INDEX\_NAME} is used for the index type\\ \hline
H5Gget\_linkval & \\ \hline
H5Gget\_num\_objs & \\ \hline
H5Gget\_objname\_by\_idx & \texttt{H5\_ITER\_DEC} is currently unsupported for the index ordering when \texttt{H5\_INDEX\_NAME} is used for the index type\\ \hline
H5Glink(2) & Currently only hard and soft link creation are supported\\ \hline
H5Gmove(2) & Refer to Notes for \texttt{H5Lmove}\\ \hline

\end{tabularx}

\textbf{Currently unsupported API calls}
\vspace{.2in} \\

\begin{tabularx}{\linewidth}{| X | >{\RaggedRight}X |}
\hline
\rowcolor{lightgray!50}%
% Center just the header row
\multicolumn{1}{| c |}{\textbf{API call}} & \multicolumn{1}{c |}{\textbf{Notes}} \\ \hline

H5Gflush & \\ \hline
H5Grefresh & \\ \hline

\end{tabularx}

\end{center}

\newpage

\subsubsection{H5L interface}

\begin{center}

\textbf{Supported API calls}
\vspace{.2in} \\

\begin{tabularx}{\linewidth}{| X | >{\RaggedRight}X |}
\hline
\rowcolor{lightgray!50}%
% Center just the header row
\multicolumn{1}{| c |}{\textbf{API call}} & \multicolumn{1}{c |}{\textbf{Notes}} \\ \hline

H5Lcreate\_hard & Reference count tracking is not currently implemented, so objects will not be removed when the last hard link pointing to them is removed\\ \hline
H5Lcreate\_soft & \\ \hline
H5Lexists & \\ \hline
H5Literate(\_by\_name) & \begin{itemize}
                             \item Restarting iteration from an index value is currently unsupported
                             \item \texttt{H5\_ITER\_DEC} is currently unsupported for the iteration order when \texttt{H5\_INDEX\_NAME} is used for the index type
                         \end{itemize}\\ \hline
H5Lvisit(\_by\_name) & \begin{itemize}
                             \item Restarting iteration from an index value is currently unsupported
                             \item \texttt{H5\_ITER\_DEC} is currently unsupported for the iteration order when \texttt{H5\_INDEX\_NAME} is used for the index type
                         \end{itemize}\\ \hline
H5Ldelete & Reference count tracking is not currently implemented, so objects will not be removed when the last hard link pointing to them is removed\\ \hline
H5Ldelete\_by\_idx & \texttt{H5\_ITER\_DEC} is currently unsupported for the index ordering when \texttt{H5\_INDEX\_NAME} is used for the index type\\ \hline
H5Lmove & Currently no support for the following properties:
\begin{itemize}
 \item LCPL
 \begin{itemize}
  \item H5Pset\_create\_intermediate\_group
 \end{itemize}
 \item LAPL
 \begin{itemize}
  \item H5Pset\_char\_encoding
  \item H5Pset\_nlinks
  \item H5Pset\_elink\_prefix
 \end{itemize}
\end{itemize}\\ \hline

\end{tabularx}

\begin{tabularx}{\linewidth}{| X | >{\RaggedRight}X |}
\hline
\rowcolor{lightgray!50}%
% Center just the header row
\multicolumn{1}{| c |}{\textbf{API call}} & \multicolumn{1}{c |}{\textbf{Notes}} \\ \hline

H5Lget\_info & Of the five fields in the \texttt{H5L\_info\_t} struct:
                                     \begin{itemize}
                                         \item \texttt{type} is set appropriately
                                         \item \texttt{corder\_valid} is set to TRUE only if link creation order tracking is enabled for the group containing the link; it is set to FALSE otherwise
                                         \item \texttt{corder} is set appropriately if link creation order tracking is enabled for the group containing the link; it is set to 0 otherwise
                                         \item \texttt{cset} is currently always set to \texttt{H5T\_CSET\_ASCII}
                                         \item \texttt{u} has member \texttt{address} or \texttt{val\_size} set appropriately based on whether the link is a hard link or not
                                     \end{itemize}\\ \hline
H5Lget\_info\_by\_idx & \texttt{H5\_ITER\_DEC} is currently unsupported for the index ordering when \texttt{H5\_INDEX\_NAME} is used for the index type\\ \hline
H5Lget\_val & \\ \hline
H5Lget\_val\_by\_idx & \texttt{H5\_ITER\_DEC} is currently unsupported for the index ordering when \texttt{H5\_INDEX\_NAME} is used for the index type\\ \hline
H5Lget\_name\_by\_idx & \texttt{H5\_ITER\_DEC} is currently unsupported for the index ordering when \texttt{H5\_INDEX\_NAME} is used for the index type\\ \hline
H5Lcopy & Currently no support for the following properties:
\begin{itemize}
 \item LCPL
 \begin{itemize}
  \item H5Pset\_create\_intermediate\_group
 \end{itemize}
 \item LAPL
 \begin{itemize}
  \item H5Pset\_char\_encoding
  \item H5Pset\_nlinks
  \item H5Pset\_elink\_prefix
 \end{itemize}
\end{itemize}\\ \hline

\end{tabularx}

\newpage

\textbf{Currently unsupported API calls}
\vspace{.2in} \\

\begin{tabularx}{\linewidth}{| X | >{\RaggedRight}X |}
\hline
\rowcolor{lightgray!50}%
% Center just the header row
\multicolumn{1}{| c |}{\textbf{API call}} & \multicolumn{1}{c |}{\textbf{Notes}} \\ \hline

H5Lcreate\_external & \\ \hline
H5Lcreate\_ud & \\ \hline

\end{tabularx}

\end{center}

\newpage

\subsubsection{H5O interface}

\begin{center}

\textbf{Supported API calls}
\vspace{.2in} \\

\begin{tabularx}{\linewidth}{| X | >{\RaggedRight}X |}
\hline
\rowcolor{lightgray!50}%
% Center just the header row
\multicolumn{1}{| c |}{\textbf{API call}} & \multicolumn{1}{c |}{\textbf{Notes}} \\ \hline

H5Oopen & \\ \hline
H5Oopen\_by\_addr & \\ \hline
H5Oopen\_by\_idx & \texttt{H5\_ITER\_DEC} is currently unsupported for the index ordering when \texttt{H5\_INDEX\_NAME} is used for the index type\\ \hline
H5Oclose & \\ \hline
H5Olink & \\ \hline
H5Oexists\_by\_name & \\ \hline
H5Ovisit(1/2) & \texttt{H5\_ITER\_DEC} is currently unsupported for the index ordering when \texttt{H5\_INDEX\_NAME} is used for the index type\\ \hline
H5Ovisit\_by\_name(1/2) & \texttt{H5\_ITER\_DEC} is currently unsupported for the index ordering when \texttt{H5\_INDEX\_NAME} is used for the index type\\ \hline
H5Ocopy & Currently no support for the following properties: \begin{itemize}
                                                                 \setlength{\itemindent}{-1em}
                                                                 \item OCpyPL
                                                                 \begin{itemize}
                                                                     \setlength{\itemindent}{-2.5em}
                                                                     \item {\small\texttt{H5O\_COPY\_EXPAND\_EXT\_LINK\_FLAG}}
                                                                     \item {\small\texttt{H5O\_COPY\_EXPAND\_REFERENCE\_FLAG}}
                                                                     \item {\small\texttt{H5O\_COPY\_MERGE\_COMMITTED\_DTYPE\_FLAG}}
                                                                 \end{itemize}
                                                                 \item LCPL
                                                                 \begin{itemize}
                                                                     \setlength{\itemindent}{-2.5em}
                                                                     \item H5Pset\_create\_intermediate\_group
                                                                 \end{itemize}
                                                             \end{itemize}\\ \hline

\end{tabularx}

\textbf{Currently unsupported API calls}
\vspace{.2in} \\

\begin{tabularx}{\linewidth}{| X | >{\RaggedRight}X |}
\hline
\rowcolor{lightgray!50}%
% Center just the header row
\multicolumn{1}{| c |}{\textbf{API call}} & \multicolumn{1}{c |}{\textbf{Notes}} \\ \hline

H5Oflush & \\ \hline
H5Orefresh & \\ \hline
H5Oincr\_refcount & \\ \hline
H5Odecr\_refcount & \\ \hline

\end{tabularx}

\end{center}

\newpage

\subsubsection{H5R interface}

\begin{center}

\textbf{Supported API calls\footnotemark[1]}
\vspace{.2in} \\

\begin{tabularx}{\linewidth}{| X | >{\RaggedRight}X |}
\hline
\rowcolor{lightgray!50}%
% Center just the header row
\multicolumn{1}{| c |}{\textbf{API call}} & \multicolumn{1}{c |}{\textbf{Notes}} \\ \hline

 & \\ \hline

\end{tabularx}

\textbf{Currently unsupported API calls}
\vspace{.2in} \\

\begin{tabularx}{\linewidth}{| X | >{\RaggedRight}X |}
\hline
\rowcolor{lightgray!50}%
% Center just the header row
\multicolumn{1}{| c |}{\textbf{API call}} & \multicolumn{1}{c |}{\textbf{Notes}} \\ \hline

H5Rcreate & \\ \hline
H5Rdereference(1/2) & Causes an assertion failure\\ \hline
H5Rget\_name & \\ \hline
H5Rget\_obj\_type(1/2) & \\ \hline
H5Rget\_region & \\ \hline

\end{tabularx}

\end{center}

\footnotetext[1]{New reference API will be supported by end of Q4 2019.}

\newpage

\subsubsection{H5T interface}

\begin{center}

\textbf{Supported API calls}
\vspace{.2in} \\

\begin{tabularx}{\linewidth}{| X | >{\RaggedRight}X |}
\hline
\rowcolor{lightgray!50}%
% Center just the header row
\multicolumn{1}{| c |}{\textbf{API call}} & \multicolumn{1}{c |}{\textbf{Notes}} \\ \hline

H5Tcommit(1/2) & \\ \hline
H5Tcommit\_anon & \\ \hline
H5Topen(1/2) & \\ \hline
H5Tclose & \\ \hline
H5Tget\_create\_plist & \\ \hline

\end{tabularx}

\textbf{Currently unsupported API calls}
\vspace{.2in} \\

\begin{tabularx}{\linewidth}{| X | >{\RaggedRight}X |}
\hline
\rowcolor{lightgray!50}%
% Center just the header row
\multicolumn{1}{| c |}{\textbf{API call}} & \multicolumn{1}{c |}{\textbf{Notes}} \\ \hline

H5Tflush & \\ \hline
H5Trefresh & \\ \hline

\end{tabularx}

\end{center}

\newpage

\subsection{Known Limitations}

The following sections outline the known current limitations of the \dvc{}.

\subsubsection{Limitations in regards to the HDF5 API}

\begin{itemize}
 \item When performing dataset I/O, the selections in the memory and file dataspaces must currently be the same shape; using selections of different shapes will result in an error stating this.\footnotemark[1]
 \item If an application abnormally exits, the \dvc{} currently leaves the file in an unusable state. Currently, the only way to re-use the same filename after an application interruption is to use a new \acrshort{daos} pool. This issue will be resolved when rollback
to a previous snapshot is supported.
\end{itemize}

\subsubsection{Limitations in regards to DAOS}

\begin{itemize}
 \item Following the previous point about application abnormal exits, as \acrshort{daos} does not currently support forced container deletion, trying to overwrite an existing \acrshort{hdf5} file using the \texttt{H5F\_ACC\_TRUNC} flag when the file was left in an unusable state will fail; the error \textit{``can't destroy container: generic I/O error (DER\_IO)''} will be returned.
\item No support for conditional key insert/remove.\todo{explain}
\item There is currently no support for distributed transactions.
\end{itemize}

\footnotetext[1]{Will be supported by end of Q4 2019.}
\end{document}
