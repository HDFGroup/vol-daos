\documentclass[../users_guide.tex]{subfiles}
 
\begin{document}

\section{Reference Manual}
\label{apdx:ref_manual}

\subsection{H5daos\_init}
\label{ref:h5daos_init}

\paragraph{Synopsis:}
\begin{flushleft}%
\begin{minted}[breaklines=true,fontsize=\small]{hdf5-c-lexer.py:HDF5CLexer -x}
herr_t H5daos_init(MPI_Comm pool_comm, uuid_t pool_uuid, const char *pool_grp,
    const char *pool_svcl);
\end{minted}
\end{flushleft}%

\paragraph{Purpose:}
\begin{flushleft}%
Initialize the DAOS VOL connector.
\end{flushleft}%

\paragraph{Description:}
\begin{flushleft}%
\texttt{H5daos\_init} initializes the VOL connector by connecting to the pool
and registering the connector with the library. \texttt{pool\_comm}
identifies the communicator used to connect to the DAOS pool.  This should
include all processes that will participate in I/O. This call is collective
across \texttt{pool\_comm}.
\end{flushleft}%

\paragraph{Parameters:}
\begin{flushleft}%
 \begin{tabular}{ll}%
   \texttt{MPI\_Comm pool\_comm} & IN: Communicator used for connecting to the pool \\
   \texttt{uuid\_t pool\_uuid} & IN: UUID to identify a pool \\
   \texttt{const char *pool\_grp} & IN: Process set name of the DAOS servers managing the pool \\
   \texttt{const char *pool\_svcl} & IN: Comma-separated list of pool service replica ranks \\
 \end{tabular}%
\end{flushleft}%

\paragraph{Returns:}
\begin{flushleft}%
Returns a non-negative value if successful; otherwise returns a negative value.
\end{flushleft}%

%%%%%%%%%%%%%%%%%%%%%%%%%%%%%%%%%%%%%%%%%%%%%%%%%%%%%%%%%%%%%%%%%%%%%%%%%%%%%%
\newpage
\subsection{H5daos\_term}
\label{ref:h5daos_term}

\paragraph{Synopsis:}
\begin{flushleft}%
\begin{minted}[breaklines=true,fontsize=\small]{hdf5-c-lexer.py:HDF5CLexer -x}
herr_t H5daos_term(void);
\end{minted}
\end{flushleft}%

\paragraph{Purpose:}
\begin{flushleft}%
Terminate the DAOS VOL connector.
\end{flushleft}%

\paragraph{Description:}
\begin{flushleft}%
\texttt{H5daos\_term} terminates the DAOS VOL connector.
\end{flushleft}%

\paragraph{Parameters:}
\begin{flushleft}%
None.
\end{flushleft}%

\paragraph{Returns:}
\begin{flushleft}%
Returns a non-negative value if successful; otherwise returns a negative value.
\end{flushleft}%

%%%%%%%%%%%%%%%%%%%%%%%%%%%%%%%%%%%%%%%%%%%%%%%%%%%%%%%%%%%%%%%%%%%%%%%%%%%%%%
\newpage
\subsection{H5Pset\_fapl\_daos}
\label{ref:h5pset_fapl_daos}

\paragraph{Synopsis:}
\begin{flushleft}%
\begin{minted}[breaklines=true,fontsize=\small]{hdf5-c-lexer.py:HDF5CLexer -x}
herr_t H5Pset_fapl_daos(hid_t fapl_id, MPI_Comm comm, MPI_Info info);
\end{minted}
\end{flushleft}%

\paragraph{Purpose:}
\begin{flushleft}%
Set the file access property list to use the DAOS VOL connector.
\end{flushleft}%

\paragraph{Description:}
\begin{flushleft}%
\texttt{H5Pset\_fapl\_daos} modifies the file access property list to use the
DAOS VOL connector. \texttt{file\_comm} and
\texttt{file\_info} identify the communicator and info object used to
coordinate actions on file create, open, flush, and close.
\end{flushleft}%

\paragraph{Parameters:}
\begin{flushleft}%
 \begin{tabular}{ll}%
   \texttt{hid\_t fapl\_id} & IN: File access property list ID \\
   \texttt{MPI\_Comm file\_comm} & IN: MPI Communicator \\
   \texttt{MPI\_Info file\_info} & IN: MPI Info \\
 \end{tabular}%
\end{flushleft}%

\paragraph{Returns:}
\begin{flushleft}%
Returns a non-negative value if successful; otherwise returns a negative value.
\end{flushleft}%

%%%%%%%%%%%%%%%%%%%%%%%%%%%%%%%%%%%%%%%%%%%%%%%%%%%%%%%%%%%%%%%%%%%%%%%%%%%%%%
\newpage
\subsection{H5daos\_set\_all\_ind\_metadata\_ops}
\label{ref:h5daos_set_all_ind_metadata_ops}

\paragraph{Synopsis:}
\begin{flushleft}%
\begin{minted}[breaklines=true,fontsize=\small]{hdf5-c-lexer.py:HDF5CLexer -x}
herr_t H5daos_set_all_ind_metadata_ops(hid_t accpl_id, hbool_t
    is_independent);
\end{minted}
\end{flushleft}%

\paragraph{Purpose:}
\begin{flushleft}%
Sets the I/O mode for metadata read/write operations in the access property list \texttt{accpl\_id}.

When engaging in parallel I/O with the DAOS VOL connector, all metadata read operations are
independent and all metadata write operations are collective by default. If \texttt{is\_independent} is
specified as \texttt{TRUE}, this property indicates that the DAOS VOL connector will perform all metadata
read and write operations independently.

If this property is set to \texttt{TRUE} on a file access property list that is used in creating or opening
a file, the DAOS VOL connector will assume that all metadata read and write operations issued on that file
identifier should be issued independently from all ranks, irrespective of the individual setting for a
particular operation.

Alternatively, a user may wish to avoid setting this property globally on the file access property list
and individually set it on particular object access property lists (dataset, group, link, datatype, attribute
access property lists) for certain operations instead. This will indicate that only the operations issued
with such an access property list will perform metadata I/O independently, whereas other operations may
perform metadata I/O collectively.  

\end{flushleft}%

\paragraph{Description:}
\begin{flushleft}%
\texttt{H5daos\_set\_all\_ind\_metadata\_ops} modifies the access property list to indicate
that metadata I/O operations should be performed independently.
\end{flushleft}%

\paragraph{Parameters:}
\begin{flushleft}%
 \begin{tabular}{lp{0.8\linewidth}}%
   \texttt{hid\_t accpl\_id} & IN: File, group, dataset, datatype, link or attribute access property list ID \\
   \texttt{hbool\_t is\_independent} & IN: Boolean value indicating whether metadata I/O operations should be \
   performed independently (\texttt{TRUE}) or should be allowed to be performed collectively (\texttt{FALSE}). \\
 \end{tabular}%
\end{flushleft}%

\paragraph{Returns:}
\begin{flushleft}%
Returns a non-negative value if successful; otherwise returns a negative value.
\end{flushleft}%

%%%%%%%%%%%%%%%%%%%%%%%%%%%%%%%%%%%%%%%%%%%%%%%%%%%%%%%%%%%%%%%%%%%%%%%%%%%%%%
\newpage
\subsection{H5daos\_get\_all\_ind\_metadata\_ops}
\label{ref:h5daos_get_all_ind_metadata_ops}

\paragraph{Synopsis:}
\begin{flushleft}%
\begin{minted}[breaklines=true,fontsize=\small]{hdf5-c-lexer.py:HDF5CLexer -x}
herr_t H5daos_get_all_ind_metadata_ops(hid_t accpl_id, hbool_t
    *is_independent);
\end{minted}
\end{flushleft}%

\paragraph{Purpose:}
\begin{flushleft}%
Retrieves the independent metadata I/O setting from the access property list \texttt{accpl\_id}.
\end{flushleft}%

\paragraph{Description:}
\begin{flushleft}%
\texttt{H5daos\_get\_all\_ind\_metadata\_ops} retrieves the independent metadata I/O setting
from the access property list \texttt{accpl\_id}.
\end{flushleft}%

\paragraph{Parameters:}
\begin{flushleft}%
 \begin{tabular}{lp{0.8\linewidth}}%
   \texttt{hid\_t accpl\_id} & IN: File, group, dataset, datatype, link or attribute access property list ID \\
   \texttt{hbool\_t *is\_independent} & OUT: Pointer to a Boolean value to be set, indicating whether metadata
   I/O is performed independently or is allowed to be performed collectively. \\
 \end{tabular}%
\end{flushleft}%

\paragraph{Returns:}
\begin{flushleft}%
Returns a non-negative value if successful; otherwise returns a negative value.
\end{flushleft}%

\end{document}
