\documentclass[../users_guide.tex]{subfiles}
 
\begin{document}

\section{Reference Manual}
\label{apdx:ref_manual}

\subsection{H5Pset\_fapl\_daos}
\label{ref:h5pset_fapl_daos}

\paragraph{Synopsis:}
\begin{flushleft}%
\begin{minted}[breaklines=true,fontsize=\small]{hdf5-c-lexer.py:HDF5CLexer -x}
herr_t H5Pset_fapl_daos(hid_t fapl_id, const char *pool, const char *daos_sys);
\end{minted}
\end{flushleft}%

\paragraph{Purpose:}
\begin{flushleft}%
Set the file access property list to use the DAOS VOL connector.
\end{flushleft}%

\paragraph{Description:}
\begin{flushleft}%
\texttt{H5Pset\_fapl\_daos} modifies the file access property list to use the
DAOS VOL connector.

\texttt{pool} identifies the label or UUID of the DAOS pool to connect to.

\texttt{pool\_grp} identifies the DAOS system name to use when connecting to
the DAOS pool. This may be NULL, in which case a default system name is used.
\end{flushleft}%

\paragraph{Parameters:}
\begin{flushleft}%
 \begin{tabular}{ll}%
   \texttt{hid\_t fapl\_id} & IN: File access property list ID \\
   \texttt{const char *pool} & IN: DAOS pool UUID or Label \\
   \texttt{const char *daos\_sys} & IN: DAOS System name where pool exists \\
 \end{tabular}%
\end{flushleft}%

\paragraph{Returns:}
\begin{flushleft}%
Returns a non-negative value if successful; otherwise returns a negative value.
\end{flushleft}%

%%%%%%%%%%%%%%%%%%%%%%%%%%%%%%%%%%%%%%%%%%%%%%%%%%%%%%%%%%%%%%%%%%%%%%%%%%%%%%
\newpage
\subsection{H5daos\_set\_prop}
\label{ref:h5daos_set_prop}

\paragraph{Synopsis:}
\begin{flushleft}%
\begin{minted}[breaklines=true,fontsize=\small]{hdf5-c-lexer.py:HDF5CLexer -x}
herr_t H5daos_set_prop(hid_t fcpl_id, const char *prop_str);
\end{minted}
\end{flushleft}%

\paragraph{Purpose:}
\begin{flushleft}%
Sets DAOS property entries for the DAOS container corresponding to the HDF5 file created. The DAOS
property entries include the Redundancy Factor of the container, container label, etc.

The property string should be of the format: 
prop_entry_name1:value1;prop_entry_name2:value2;prop_entry_name3:value3;

\end{flushleft}%

\paragraph{Description:}
\begin{flushleft}%
\texttt{H5daos\_set\_prop} modifies the file creation property list to set DAOS
container property entries.
\end{flushleft}%

\paragraph{Parameters:}
\begin{flushleft}%
 \begin{tabular}{lp{0.8\linewidth}}%
   \texttt{hid\_t fcpl\_id} & IN: File creation property list ID \\
   \texttt{const char *prop\_str} & IN: String value of property entries and values\\
 \end{tabular}%
\end{flushleft}%

\paragraph{Returns:}
\begin{flushleft}%
Returns a non-negative value if successful; otherwise returns a negative value.
\end{flushleft}%

%%%%%%%%%%%%%%%%%%%%%%%%%%%%%%%%%%%%%%%%%%%%%%%%%%%%%%%%%%%%%%%%%%%%%%%%%%%%%%
\newpage
\subsection{H5daos\_set\_all\_ind\_metadata\_ops}
\label{ref:h5daos_set_all_ind_metadata_ops}

\paragraph{Synopsis:}
\begin{flushleft}%
\begin{minted}[breaklines=true,fontsize=\small]{hdf5-c-lexer.py:HDF5CLexer -x}
herr_t H5daos_set_all_ind_metadata_ops(hid_t accpl_id,
                                       hbool_t is_independent);
\end{minted}
\end{flushleft}%

\paragraph{Purpose:}
\begin{flushleft}%
Sets the I/O mode for metadata read/write operations in the access property list \texttt{accpl\_id}.

When engaging in parallel I/O with the DAOS VOL connector, all metadata read operations are
independent and all metadata write operations are collective by default. If \texttt{is\_independent} is
specified as \texttt{TRUE}, this property indicates that the DAOS VOL connector will perform all metadata
read and write operations independently.

If this property is set to \texttt{TRUE} on a file access property list that is used in creating or opening
a file, the DAOS VOL connector will assume that all metadata read and write operations issued on that file
identifier should be issued independently from all ranks, irrespective of the individual setting for a
particular operation.

Alternatively, a user may wish to avoid setting this property globally on the file access property list
and individually set it on particular object access property lists (dataset, group, link, datatype, attribute
access property lists) for certain operations instead. This will indicate that only the operations issued
with such an access property list will perform metadata I/O independently, whereas other operations may
perform metadata I/O collectively.  

\end{flushleft}%

\paragraph{Description:}
\begin{flushleft}%
\texttt{H5daos\_set\_all\_ind\_metadata\_ops} modifies the access property list to indicate
that metadata I/O operations should be performed independently.
\end{flushleft}%

\paragraph{Parameters:}
\begin{flushleft}%
 \begin{tabular}{lp{0.8\linewidth}}%
   \texttt{hid\_t accpl\_id} & IN: File, group, dataset, datatype, link or attribute access property list ID \\
   \texttt{hbool\_t is\_independent} & IN: Boolean value indicating whether metadata I/O operations should be \
   performed independently (\texttt{TRUE}) or should be allowed to be performed collectively (\texttt{FALSE}). \\
 \end{tabular}%
\end{flushleft}%

\paragraph{Returns:}
\begin{flushleft}%
Returns a non-negative value if successful; otherwise returns a negative value.
\end{flushleft}%

%%%%%%%%%%%%%%%%%%%%%%%%%%%%%%%%%%%%%%%%%%%%%%%%%%%%%%%%%%%%%%%%%%%%%%%%%%%%%%
\newpage
\subsection{H5daos\_get\_all\_ind\_metadata\_ops}
\label{ref:h5daos_get_all_ind_metadata_ops}

\paragraph{Synopsis:}
\begin{flushleft}%
\begin{minted}[breaklines=true,fontsize=\small]{hdf5-c-lexer.py:HDF5CLexer -x}
herr_t H5daos_get_all_ind_metadata_ops(hid_t accpl_id,
                                       hbool_t *is_independent);
\end{minted}
\end{flushleft}%

\paragraph{Purpose:}
\begin{flushleft}%
Retrieves the independent metadata I/O setting from the access property list \texttt{accpl\_id}.
\end{flushleft}%

\paragraph{Description:}
\begin{flushleft}%
\texttt{H5daos\_get\_all\_ind\_metadata\_ops} retrieves the independent metadata I/O setting
from the access property list \texttt{accpl\_id}.
\end{flushleft}%

\paragraph{Parameters:}
\begin{flushleft}%
 \begin{tabular}{lp{0.8\linewidth}}%
   \texttt{hid\_t accpl\_id} & IN: File, group, dataset, datatype, link or attribute access property list ID \\
   \texttt{hbool\_t *is\_independent} & OUT: Pointer to a Boolean value to be set, indicating whether metadata
   I/O is performed independently or is allowed to be performed collectively. \\
 \end{tabular}%
\end{flushleft}%

\paragraph{Returns:}
\begin{flushleft}%
Returns a non-negative value if successful; otherwise returns a negative value.
\end{flushleft}%

\end{document}
