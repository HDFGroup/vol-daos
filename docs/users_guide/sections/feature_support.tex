\documentclass[../users_guide.tex]{subfiles}
 
\begin{document}

\section{HDF5 API Support}

\subsection{Feature Specific Support}

The following sections serve to illustrate the \dvc{}'s support for features in \acrshort{hdf5}, as well as to highlight any differences between the expected behavior of an \acrshort{hdf5} feature versus the actual behavior as implemented by the VOL connector.

\subsubsection{Attribute Features}

\begin{tabularx}{\linewidth}{| X | X | X | X | >{\RaggedRight}X |}
\hline
\rowcolor{lightgray!50}%
% Center just the header row
\multicolumn{3}{| c |}{\textbf{Feature}} & \multicolumn{1}{c |}{\textbf{Supported?}} & \multicolumn{1}{c |}{\textbf{Notes}} \\ \hline

\multirow[c]{3}{\linewidth}{Dataspace} & \multirow[c]{3}{\linewidth}{Dimensionality} & H5S\_NULL & Yes & \\ \cline{3-4}
& & H5S\_SCALAR & Yes & \\ \cline{3-4}
& & SIMPLE & Yes & \\ \cline{3-4} \hline

\multicolumn{2}{| l |}{\multirow[c]{6}{*}[-90pt]{Datatype}} & Atomic & Yes & \\ [30pt] \cline{3-4}
\multicolumn{2}{| l |}{} & Compound & Yes & \\ [30pt] \cline{3-4}
\multicolumn{2}{| l |}{} & Variable-length & Yes & \\ [30pt] \cline{3-4}
\multicolumn{2}{| l |}{} & Array & Yes & \\ [30pt] \cline{3-4}
\multicolumn{2}{| l |}{} & Opaque & Yes & \\ [30pt] \cline{3-4}
\multicolumn{2}{| l |}{} & Reference & Yes & \\ [30pt] \cline{3-4}
\hline

\multirow[c]{2}{\linewidth}{Properties} & \multirow[c]{2}{\linewidth}{Name Encoding} & ASCII & Yes & \\ \cline{3-5}
& & UTF-8 & No & \\ \cline{3-4} \hline

\end{tabularx}

\newpage

\subsubsection{Dataset Features}

\begin{tabularx}{\linewidth}{| X | X | X | X | >{\RaggedRight}X |}
\hline
\rowcolor{lightgray!50}%
% Center just the header row
\multicolumn{3}{| c |}{\textbf{Feature}} & \multicolumn{1}{c |}{\textbf{Supported?}} & \multicolumn{1}{c |}{\textbf{Notes}} \\ \hline

\multirow[c]{7}{\linewidth}[-45pt]{Dataspace} & \multirow[c]{3}{\linewidth}{Dimensionality} & H5S\_NULL & Yes & \\ \cline{3-4}
& & H5S\_SCALAR & Yes & \\ \cline{3-4}
& & SIMPLE & Yes & \\ \cline{3-4} \cline{2-5}
& \multirow[c]{4}{\linewidth}[-55pt]{Selection Type} & NONE & Yes & \\ [28pt] \cline{3-4}
& & H5S\_ALL & Yes & \\ [28pt] \cline{3-4}
& & Hyperslab Selection & Yes & \\ [28pt] \cline{3-4}
& & Point Selection & Yes & \\ [28pt] \cline{3-4} \hline

\multicolumn{2}{| l |}{\multirow[c]{6}{*}[-90pt]{Datatype}} & Atomic & Yes & \\ [30pt] \cline{3-4}
\multicolumn{2}{| l |}{} & Compound & Yes & \\ [30pt] \cline{3-4}
\multicolumn{2}{| l |}{} & Variable-length & Yes & \\ [30pt] \cline{3-4}
\multicolumn{2}{| l |}{} & Array & Yes & \\ [30pt] \cline{3-4}
\multicolumn{2}{| l |}{} & Opaque & Yes & \\ [30pt] \cline{3-4}
\multicolumn{2}{| l |}{} & Reference & Yes & \\ [30pt] \cline{3-4}
\hline

\end{tabularx}

\newpage

\begin{tabularx}{\linewidth}{| X | X | X | X | >{\RaggedRight}X |}
\hline
\rowcolor{lightgray!50}%
% Center just the header row
\multicolumn{3}{| c |}{\textbf{Feature}} & \multicolumn{1}{c |}{\textbf{Supported?}} & \multicolumn{1}{c |}{\textbf{Notes}} \\ \hline

\multirow[c]{9}{\linewidth}[-80pt]{Properties} & \multirow[c]{5}{\linewidth}[-35pt]{Storage Properties (creation)} & Compact & No & Setting is ignored; stored as contiguous. \\ \cline{3-5}
& & External & No & Setting is ignored; stored as contiguous. \\ \cline{3-5}
& & Contiguous & Yes & Default storage type. \\ \cline{3-5}
& & Chunked & Yes & \\ \cline{3-5}
& & VDS & No & The VDS feature is not currently planned to be supported.\\ \cline{2-5}
& \multirow[c]{4}{\linewidth}[-45pt]{Other Properties (creation)} & Attribute Creation Order & Yes & In order to work correctly, the attribute creation order feature requires that the dataset is touched collectively or that the application otherwise ensures no concurrent access to the dataset. This restriction may be removed in the future. \\ \cline{3-5}
& & Fill Value & Yes & \\ \cline{3-5}
& & Filters & No & HDF5 does not expose any public APIs for working with the filter pipeline; however, this feature may be supported in the future. \\ \cline{3-5}
& & Storage Allocation Time & N/A & \\ \hline

\end{tabularx}

\newpage

\begin{tabularx}{\linewidth}{| X | X | X | X | >{\RaggedRight}X |}
\hline
\rowcolor{lightgray!50}%
% Center just the header row
\multicolumn{3}{| c |}{\textbf{Feature}} & \multicolumn{1}{c |}{\textbf{Supported?}} & \multicolumn{1}{c |}{\textbf{Notes}} \\ \hline

\multirow[c]{4}{\linewidth}[-55pt]{Properties (cont.)} & \multirow[c]{3}{\linewidth}[-50pt]{Access Properties} & Chunk cache & No & HDF5 does not expose any public APIs for implementing a chunk cache for arbitrary VOL connectors; however, this feature may be supported in the future. \\ \cline{3-5}
& & VDS views and printf & No & The VDS feature is not currently planned to be supported.\\ \cline{3-5}
& & MPI-I/O Collective Metadata Ops & Yes & By default, all metadata operations are collective for writes and independent for reads.\footnotemark[1] \\ \cline{2-5}
& \multirow[c]{1}{\linewidth}[-10pt]{Transfer Properties} & MPI-I/O Independent or Collective I/O mode & N/A & \\ \hline

\end{tabularx}

\footnotetext[1]{Independent metadata writes will be supported in the future.}

\subsubsection{File Features}

\begin{tabularx}{\linewidth}{| X | X | X | X | >{\RaggedRight}X |}
\hline
\rowcolor{lightgray!50}%
% Center just the header row
\multicolumn{3}{| c |}{\textbf{Feature}} & \multicolumn{1}{c |}{\textbf{Supported?}} & \multicolumn{1}{c |}{\textbf{Notes}} \\ \hline

\multicolumn{2}{| l |}{\multirow[c]{2}{*}{File creation flags}} & \multicolumn{1}{l |}{H5F\_ACC\_TRUNC} & Yes & \multirow[c]{4}{\linewidth}{The file creation flags behave as for native HDF5.} \\ \cline{3-4}
\multicolumn{2}{| l |}{} & \multicolumn{1}{l |}{H5F\_ACC\_EXCL} & Yes & \\ \cline{1-4}
\multicolumn{2}{| l |}{\multirow[c]{2}{*}{File opening flags}} & \multicolumn{1}{l |}{H5F\_ACC\_RDWR} & Yes & \\ \cline{3-4}
\multicolumn{2}{| l |}{} & \multicolumn{1}{l |}{H5F\_ACC\_RDONLY} & Yes & \\ \hline

\multirow[c]{8}{\linewidth}[-100pt]{Properties} & \multirow[c]{1}{\linewidth}[-70pt]{Creation Properties} & Attribute Creation Order & Yes & In order to work correctly, the attribute creation order feature requires that the dataset is touched collectively or that the application otherwise ensures no concurrent access to the dataset. This restriction may be removed in the future. The rest of the file creation properties are related to the native HDF5-specific file format. \\ \cline{2-5}

& \multirow[c]{7}{\linewidth}{Access Properties (Drivers)} & SEC2 Driver & N/A & \multirow[c]{4}{\linewidth}{These drivers are applicable to native HDF5 only.} \\ \cline{3-4}
& & Family Driver & N/A & \\ \cline{3-4}
& & Split Driver & N/A & \\ \cline{3-4}
& & Multi Driver & N/A & \\ \cline{3-4}
& & Core Driver & N/A & \\ \cline{3-4}
& & Log Driver & N/A & \\ \cline{3-5}
& & MPI-I/O & Yes & This property just indicates parallel access to the file; it doesn't use HDF5 MPI I/O driver underneath. \\ \hline

\end{tabularx}

\newpage

\begin{tabularx}{\linewidth}{| X | X | X | X | >{\RaggedRight}X |}
\hline
\rowcolor{lightgray!50}%
% Center just the header row
\multicolumn{3}{| c |}{\textbf{Feature}} & \multicolumn{1}{c |}{\textbf{Supported?}} & \multicolumn{1}{c |}{\textbf{Notes}} \\ \hline

\multirow[c]{10}{\linewidth}[-100pt]{Properties (cont.)} & \multirow[c]{8}{\linewidth}[-60pt]{Access Properties (Other)} & MPI-I/O Collective Metadata Ops & Yes & By default, all metadata operations are collective for writes and independent for reads.\footnotemark[1] \\ \cline{3-5}

& & User block & N/A & \\ \cline{3-5}
& & Chunk Cache & No & HDF5 does not expose any public APIs for implementing a chunk cache for arbitrary VOL connectors; however, this feature may be supported in the future. \\ \cline{3-5}
& & Object flushing callbacks & N/A & \\ \cline{3-5}
& & File closing degree & N/A & \\ \cline{3-5}
& & Evict on close & N/A & \\ \cline{3-5}
& & Sieve buffer size for partial I/O & No & HDF5 does not expose any public APIs for implementing a partial I/O sieve buffer for arbitrary VOL connectors; however, this feature may be supported in the future. \\ \cline{3-5}
& & File Image & N/A & \\ \hline

\end{tabularx}

\footnotetext[1]{Independent metadata writes will be supported in the future.}

\subsubsection{Group Features}

\begin{tabularx}{\linewidth}{| X | X | >{\RaggedRight}X | X | >{\RaggedRight}X |}
\hline
\rowcolor{lightgray!50}%
% Center just the header row
\multicolumn{3}{| c |}{\textbf{Feature}} & \multicolumn{1}{c |}{\textbf{Supported?}} & \multicolumn{1}{c |}{\textbf{Notes}} \\ \hline

\multirow[c]{4}{\linewidth}[-120pt]{Properties} & \multirow[c]{3}{\linewidth}[-100pt]{Creation Properties} & Link Creation Order & Yes & In order to work correctly, the link creation order feature requires that the parent group is touched collectively or that the application otherwise ensures no concurrent access to the group. This restriction may be removed in the future. \\ \cline{3-5}
& & Attribute Creation Order & Yes & In order to work correctly, the attribute creation order feature requires that the dataset is touched collectively or that the application otherwise ensures no concurrent access to the dataset. This restriction may be removed in the future. \\ \cline{3-5}
& & Other Properties & N/A & These properties are related to the native HDF5-specific file format. \\ \cline{2-5}
& \multirow[c]{1}{\linewidth}[-35pt]{Access Properties} & MPI-I/O Collective Metadata Ops & Yes & By default, all metadata operations are collective for writes and independent for reads.\footnotemark[1] \\ \hline

\end{tabularx}

\footnotetext[1]{Independent metadata writes will be supported in the future.}

\newpage

\subsection{API Specific Support}

The following sections serve to illustrate the \dvc{}'s support for the \acrshort{hdf5} API, as well as to highlight any differences between the expected behavior of an \acrshort{hdf5} API call versus the actual behavior as implemented by the VOL connector. If a particular \acrshort{hdf5} API call does not appear among these tables, it is most likely a native \acrshort{hdf5}-specific API call which cannot be implemented by non-native \acrshort{hdf5} VOL connectors. These types of API calls are listed among the tables in Appendix~\ref{apdx:native_calls}.

\newpage

\subsubsection{H5A interface}

\begin{center}

\textbf{Supported API calls}
\vspace{.2in} \\

\begin{tabularx}{\linewidth}{| X | >{\RaggedRight}X |}
\hline
\rowcolor{lightgray!50}%
% Center just the header row
\multicolumn{1}{| c |}{\textbf{API call}} & \multicolumn{1}{c |}{\textbf{Notes}} \\ \hline

H5Acreate(1/2) & \\ \hline
H5Acreate\_by\_name & \\ \hline
H5Aopen(\_by\_name/\_by\_idx) & For H5Aopen\_by\_idx, \texttt{H5\_ITER\_DEC} is currently unsupported for the index ordering when \texttt{H5\_INDEX\_NAME} is used for the index type\\ \hline
H5Aopen\_idx & Deprecated in favor of H5A\_open\_by\_idx\\ \hline
H5Aopen\_name & Deprecated in favor of H5A\_open\_by\_name\\ \hline
H5Awrite & \\ \hline
H5Aread & \\ \hline
H5Aclose & \\ \hline
H5Aiterate(2) & \begin{itemize}
                    \item Restarting iteration from an index value is currently unsupported\footnotemark[1]
                    \item \texttt{H5\_ITER\_DEC} is currently unsupported for the index ordering when \texttt{H5\_INDEX\_NAME} is used for the index type
                \end{itemize}\\ \hline
H5Aiterate\_by\_name & \begin{itemize}
                           \item Restarting iteration from an index value is currently unsupported\footnotemark[1]
                           \item \texttt{H5\_ITER\_DEC} is currently unsupported for the index ordering when \texttt{H5\_INDEX\_NAME} is used for the index type
                       \end{itemize}\\ \hline
H5Aexists(\_by\_name) & \\ \hline
H5Arename(\_by\_name) & \\ \hline
H5Adelete(\_by\_name/\_by\_idx) & For H5Adelete\_by\_idx, \texttt{H5\_ITER\_DEC} is currently unsupported for the index ordering when \texttt{H5\_INDEX\_NAME} is used for the index type\\ \hline
H5Aget\_name(\_by\_idx) & For H5Aget\_name\_by\_idx, \texttt{H5\_ITER\_DEC} is currently unsupported for the index ordering when \texttt{H5\_INDEX\_NAME} is used for the index type\\ \hline
\end{tabularx}

\footnotetext[1]{Will be supported by end of Q4 2019.}

\begin{tabularx}{\linewidth}{| X | >{\RaggedRight}X |}
\hline
\rowcolor{lightgray!50}%
% Center just the header row
\multicolumn{1}{| c |}{\textbf{API call}} & \multicolumn{1}{c |}{\textbf{Notes}} \\ \hline

H5Aget\_space & \\ \hline
H5Aget\_type & \\ \hline
H5Aget\_info(\_by\_name/\_by\_idx) & Of the four fields in the \texttt{H5A\_info\_t} struct:
                                     \begin{itemize}
                                         \item \texttt{corder\_valid} is set to TRUE only if attribute creation order tracking is enabled for the object containing the attribute; it is set to FALSE otherwise
                                         \item \texttt{corder} is set appropriately if attribute creation order tracking is enabled for the object containing the attribute; it is set to 0 otherwise
                                         \item \texttt{cset} is currently always set to \texttt{H5T\_CSET\_ASCII}
                                         \item \texttt{data\_size} is set appropriately
                                     \end{itemize}

                                     For H5Aget\_info\_by\_idx, \texttt{H5\_ITER\_DEC} is currently unsupported for the index ordering when \texttt{H5\_INDEX\_NAME} is used for the index type\\ \hline
H5Aget\_create\_plist & \\ \hline

\end{tabularx}

\textbf{Currently unsupported API calls}
\vspace{.1in} \\

\begin{tabularx}{\linewidth}{| X | >{\RaggedRight}X |}
\hline
\rowcolor{lightgray!50}%
% Center just the header row
\multicolumn{1}{| c |}{\textbf{API call}} & \multicolumn{1}{c |}{\textbf{Notes}} \\ \hline

H5Aget\_storage\_size & \texttt{H5Aget\_storage\_size} is not currently planned to be supported.\\ \hline

\end{tabularx}

\end{center}

\newpage

\subsubsection{H5D interface}

\begin{center}

\textbf{Supported API calls}
\vspace{.2in} \\

\begin{tabularx}{\linewidth}{| X | >{\RaggedRight}X |}
\hline
\rowcolor{lightgray!50}%
% Center just the header row
\multicolumn{1}{| c |}{\textbf{API call}} & \multicolumn{1}{c |}{\textbf{Notes}} \\ \hline

H5Dcreate(1/2) & \\ \hline
H5Dcreate\_anon & \\ \hline
H5Dopen(1/2) & \\ \hline
H5Dwrite & \\ \hline
H5Dread & \\ \hline
H5Dclose & \\ \hline
H5Dextend & Upon dataset shrinking, data is currently not cleared.\footnotemark[1]\\ \hline
H5Dset\_extent & Upon dataset shrinking, data is currently not cleared.\footnotemark[1]\\ \hline
H5Dget\_space & \\ \hline
H5Dget\_type & \\ \hline
H5Dget\_create\_plist & \\ \hline
H5Dget\_access\_plist & \\ \hline
H5Dget\_space\_status & Space status is currently always set to \texttt{H5D\_SPACE\_STATUS\_NOT\_ALLOCATED}\\ \hline
H5Dflush & \texttt{H5Dflush} is currently implemented as a no-op. \\ \hline
H5Drefresh & \\ \hline

\end{tabularx}

\textbf{Currently unsupported API calls}
\vspace{.2in} \\

\begin{tabularx}{\linewidth}{| X | >{\RaggedRight}X |}
\hline
\rowcolor{lightgray!50}%
% Center just the header row
\multicolumn{1}{| c |}{\textbf{API call}} & \multicolumn{1}{c |}{\textbf{Notes}} \\ \hline

H5Dget\_storage\_size & \texttt{H5Dget\_storage\_size} is not currently planned to be supported.\\ \hline

\end{tabularx}

\footnotetext[1]{Will be supported by end of Q5 2020.}

\end{center}

\newpage

\subsubsection{H5F interface}

\begin{center}

\textbf{Supported API calls}
\vspace{.2in} \\

\begin{tabularx}{\linewidth}{| X | >{\RaggedRight}X |}
\hline
\rowcolor{lightgray!50}%
% Center just the header row
\multicolumn{1}{| c |}{\textbf{API call}} & \multicolumn{1}{c |}{\textbf{Notes}} \\ \hline

H5Fcreate & \\ \hline
H5Fopen & \\ \hline
H5Freopen & \\ \hline
H5Fis\_accessible & \\ \hline
H5Fget\_create\_plist & \\ \hline
H5Fget\_access\_plist & \\ \hline
H5Fget\_intent & \\ \hline
H5Fget\_name & \\ \hline
H5Fget\_obj\_count & \\ \hline
H5Fget\_obj\_ids & \\ \hline
H5Fdelete & \\ \hline
H5Fflush & \texttt{H5Fflush} is currently implemented as a no-op.\\ \hline
H5Fclose & \\ \hline

\end{tabularx}

\textbf{Currently unsupported API calls}
\vspace{.2in} \\

\begin{tabularx}{\linewidth}{| X | >{\RaggedRight}X |}
\hline
\rowcolor{lightgray!50}%
% Center just the header row
\multicolumn{1}{| c |}{\textbf{API call}} & \multicolumn{1}{c |}{\textbf{Notes}} \\ \hline

H5Fmount & \texttt{H5Fmount} is not currently planned to be supported.\\ \hline
H5Funmount & \texttt{H5Funmount} is not currently planned to be supported.\\ \hline

\end{tabularx}

\end{center}

\newpage

\subsubsection{H5G interface}

\begin{center}

\textbf{Supported API calls}
\vspace{.2in} \\

\begin{tabularx}{\linewidth}{| X | >{\RaggedRight}X |}
\hline
\rowcolor{lightgray!50}%
% Center just the header row
\multicolumn{1}{| c |}{\textbf{API call}} & \multicolumn{1}{c |}{\textbf{Notes}} \\ \hline

H5Gcreate(1/2) & \\ \hline
H5Gcreate\_anon & \\ \hline
H5Gopen(1/2) & \\ \hline
H5Gclose & \\ \hline
H5Gunlink & \\ \hline
H5Gget\_create\_plist & \\ \hline
H5Gget\_info(\_by\_name/\_by\_idx) & Of the four fields in the \texttt{H5G\_info\_t} struct:
                                     \begin{itemize}
                                         \item \texttt{storage\_type} is always set to \texttt{H5G\_STORAGE\_TYPE\_UNKNOWN}
                                         \item \texttt{nlinks} is set appropriately
                                         \item \texttt{max\_corder} is set appropriately if link creation order is tracked for the group
                                         \item \texttt{mounted} is currently always set to \texttt{FALSE}
                                     \end{itemize}
                                     For \texttt{H5Gget\_info\_by\_idx}, \texttt{H5\_ITER\_DEC} is currently unsupported for the index ordering when \texttt{H5\_INDEX\_NAME} is used for the index type\\ \hline
H5Gget\_linkval & \\ \hline
H5Gget\_num\_objs & \\ \hline
H5Gget\_objname\_by\_idx & \texttt{H5\_ITER\_DEC} is currently unsupported for the index ordering when \texttt{H5\_INDEX\_NAME} is used for the index type\\ \hline
H5Glink(2) & Currently only hard and soft link creation are supported\footnotemark[1]\\ \hline
H5Gmove(2) & Refer to Notes for \texttt{H5Lmove}\\ \hline
H5Gflush & \texttt{H5Gflush} is currently implemented as a no-op.\\ \hline
H5Grefresh & \texttt{H5Grefresh} is currently implemented as a no-op.\\ \hline

\end{tabularx}

\footnotetext[1]{External links are not currently planned to be supported.}

\textbf{Currently unsupported API calls}
\vspace{.2in} \\

\begin{tabularx}{\linewidth}{| X | >{\RaggedRight}X |}
\hline
\rowcolor{lightgray!50}%
% Center just the header row
\multicolumn{1}{| c |}{\textbf{API call}} & \multicolumn{1}{c |}{\textbf{Notes}} \\ \hline

& \\ \hline

\end{tabularx}

\end{center}

\newpage

\subsubsection{H5L interface}

\begin{center}

\textbf{Supported API calls}
\vspace{.2in} \\

\begin{tabularx}{\linewidth}{| X | >{\RaggedRight}X |}
\hline
\rowcolor{lightgray!50}%
% Center just the header row
\multicolumn{1}{| c |}{\textbf{API call}} & \multicolumn{1}{c |}{\textbf{Notes}} \\ \hline

H5Lcreate\_hard & Reference count tracking is not currently implemented, so objects will not be removed when the last hard link pointing to them is removed\footnotemark[1]\\ \hline
H5Lcreate\_soft & \\ \hline
H5Lexists & \\ \hline
H5Literate(\_by\_name) & \begin{itemize}
                             \item Restarting iteration from an index value is currently unsupported\footnotemark[1]
                             \item \texttt{H5\_ITER\_DEC} is currently unsupported for the iteration order when \texttt{H5\_INDEX\_NAME} is used for the index type
                         \end{itemize}\\ \hline
H5Lvisit(\_by\_name) & \begin{itemize}
                             \item Restarting iteration from an index value is currently unsupported\footnotemark[1]
                             \item \texttt{H5\_ITER\_DEC} is currently unsupported for the iteration order when \texttt{H5\_INDEX\_NAME} is used for the index type
                         \end{itemize}\\ \hline
H5Ldelete & Reference count tracking is not currently implemented, so objects will not be removed when the last hard link pointing to them is removed\footnotemark[1]\\ \hline
H5Ldelete\_by\_idx & \texttt{H5\_ITER\_DEC} is currently unsupported for the index ordering when \texttt{H5\_INDEX\_NAME} is used for the index type\\ \hline

\end{tabularx}

\footnotetext[1]{Will be supported by end of Q4 2019.}

\begin{tabularx}{\linewidth}{| X | >{\RaggedRight}X |}
\hline
\rowcolor{lightgray!50}%
% Center just the header row
\multicolumn{1}{| c |}{\textbf{API call}} & \multicolumn{1}{c |}{\textbf{Notes}} \\ \hline

H5Lget\_info & Of the five fields in the \texttt{H5L\_info\_t} struct:
                                     \begin{itemize}
                                         \item \texttt{type} is set appropriately
                                         \item \texttt{corder\_valid} is set to TRUE only if link creation order tracking is enabled for the group containing the link; it is set to FALSE otherwise
                                         \item \texttt{corder} is set appropriately if link creation order tracking is enabled for the group containing the link; it is set to 0 otherwise
                                         \item \texttt{cset} is currently always set to \texttt{H5T\_CSET\_ASCII}
                                         \item \texttt{u} has member \texttt{address} or \texttt{val\_size} set appropriately based on whether the link is a hard link or not
                                     \end{itemize}\\ \hline
H5Lget\_info\_by\_idx & \texttt{H5\_ITER\_DEC} is currently unsupported for the index ordering when \texttt{H5\_INDEX\_NAME} is used for the index type\\ \hline
H5Lget\_val & \\ \hline
H5Lget\_val\_by\_idx & \texttt{H5\_ITER\_DEC} is currently unsupported for the index ordering when \texttt{H5\_INDEX\_NAME} is used for the index type\\ \hline
H5Lget\_name\_by\_idx & \texttt{H5\_ITER\_DEC} is currently unsupported for the index ordering when \texttt{H5\_INDEX\_NAME} is used for the index type\\ \hline

\end{tabularx}

\newpage

\begin{tabularx}{\linewidth}{| X | >{\RaggedRight}X |}
\hline
\rowcolor{lightgray!50}%
% Center just the header row
\multicolumn{1}{| c |}{\textbf{API call}} & \multicolumn{1}{c |}{\textbf{Notes}} \\ \hline

H5Lcopy & Currently no support for the following properties:
\begin{itemize}
 \item LAPL
 \begin{itemize}
  \item H5Pset\_nlinks
  \item H5Pset\_elink\_prefix\footnotemark[1]
 \end{itemize}
\end{itemize}\\ \hline
H5Lmove & Currently no support for the following properties:
\begin{itemize}
 \item LAPL
 \begin{itemize}
  \item H5Pset\_nlinks
  \item H5Pset\_elink\_prefix\footnotemark[1]
 \end{itemize}
\end{itemize}\\ \hline

\end{tabularx}

\textbf{Currently unsupported API calls}
\vspace{.2in} \\

\begin{tabularx}{\linewidth}{| X | >{\RaggedRight}X |}
\hline
\rowcolor{lightgray!50}%
% Center just the header row
\multicolumn{1}{| c |}{\textbf{API call}} & \multicolumn{1}{c |}{\textbf{Notes}} \\ \hline

H5Lcreate\_external & \texttt{H5Lcreate\_external} is not currently planned to be supported. As DAOS containers can contain large amounts of objects, the necessity for external links is lessened as compared to a traditional storage system. \\ \hline
H5Lcreate\_ud & \texttt{H5Lcreate\_ud} is not currently planned to be supported.\\ \hline

\end{tabularx}

\footnotetext[1]{External links are not currently planned to be supported.}

\end{center}

\newpage

\subsubsection{H5O interface}

\begin{center}

\textbf{Supported API calls}
\vspace{.2in} \\

\begin{tabularx}{\linewidth}{| X | >{\RaggedRight}X |}
\hline
\rowcolor{lightgray!50}%
% Center just the header row
\multicolumn{1}{| c |}{\textbf{API call}} & \multicolumn{1}{c |}{\textbf{Notes}} \\ \hline

H5Oopen & \\ \hline
H5Oopen\_by\_addr & \\ \hline
H5Oopen\_by\_idx & \texttt{H5\_ITER\_DEC} is currently unsupported for the index ordering when \texttt{H5\_INDEX\_NAME} is used for the index type\\ \hline
H5Oclose & \\ \hline
H5Olink & \\ \hline
H5Oexists\_by\_name & \\ \hline
H5Ovisit(1/2) & \texttt{H5\_ITER\_DEC} is currently unsupported for the index ordering when \texttt{H5\_INDEX\_NAME} is used for the index type\\ \hline
H5Ovisit\_by\_name(1/2) & \texttt{H5\_ITER\_DEC} is currently unsupported for the index ordering when \texttt{H5\_INDEX\_NAME} is used for the index type\\ \hline
H5Ocopy & Currently no support for the following properties: \begin{itemize}
                                                                 \setlength{\itemindent}{-1em}
                                                                 \item OCpyPL
                                                                 \begin{itemize}
                                                                     \setlength{\itemindent}{-2.5em}
                                                                     \item {\small\texttt{H5O\_COPY\_EXPAND\_EXT\_LINK\_FLAG}}\footnotemark[2]
                                                                     \item {\small\texttt{H5O\_COPY\_EXPAND\_REFERENCE\_FLAG}}\footnotemark[1]
                                                                     \item {\small\texttt{H5O\_COPY\_MERGE\_COMMITTED\_DTYPE\_FLAG}}
                                                                 \end{itemize}
                                                             \end{itemize}\\ \hline
H5Oflush & \texttt{H5Oflush} delegates to the appropriate \texttt{H5Xflush} routine based upon the given object's type, but that particular object's flush routine may be implemented as a no-op. \\ \hline
H5Orefresh & \texttt{H5Orefresh} delegates to the appropriate \texttt{H5Xrefresh} routine based upon the given object's type, but that particular object's refresh routine may be implemented as a no-op. \\ \hline

\end{tabularx}

\footnotetext[1]{Will be supported by end of Q5 2020.}
\footnotetext[2]{External links are not currently planned to be supported.}

\newpage

\textbf{Currently unsupported API calls}
\vspace{.2in} \\

\begin{tabularx}{\linewidth}{| X | >{\RaggedRight}X |}
\hline
\rowcolor{lightgray!50}%
% Center just the header row
\multicolumn{1}{| c |}{\textbf{API call}} & \multicolumn{1}{c |}{\textbf{Notes}} \\ \hline

H5Oincr\_refcount\footnotemark[1] & \\ \hline
H5Odecr\_refcount\footnotemark[1] & \\ \hline

\end{tabularx}

\footnotetext[1]{Will be supported by end of Q4 2019.}

\end{center}

\newpage

\subsubsection{H5R interface}

\begin{center}

\textbf{Supported API calls}
\vspace{.2in} \\

\begin{tabularx}{\linewidth}{| X | >{\RaggedRight}X |}
\hline
\rowcolor{lightgray!50}%
% Center just the header row
\multicolumn{1}{| c |}{\textbf{API call}} & \multicolumn{1}{c |}{\textbf{Notes}} \\ \hline

H5Rcreate\_object & Object lookup and retrieval of a container's info (which are needed for \texttt{H5Rcreate\_object}) are supported.\\ \hline
H5Rcreate\_region & Object lookup and retrieval of a container's info (which are needed for \texttt{H5Rcreate\_region}) are supported.\\ \hline
H5Rcreate\_attr & Object lookup and retrieval of a container's info (which are needed for \texttt{H5Rcreate\_attr}) are supported.\\ \hline
H5Ropen\_object & Object opening (which is needed for \texttt{H5Ropen\_object}) is supported.\\ \hline
H5Ropen\_region & Object opening and retrieval of a dataset's dataspace (which are needed for \texttt{H5Ropen\_region}) are supported.\\ \hline
H5Ropen\_attr & Object opening and attribute opening (which are needed for \texttt{H5Ropen\_attr}) are supported.\\ \hline
H5Rget\_obj\_type3 & \\ \hline
H5Rget\_file\_name & Retrieval of a file's name (which is needed for \texttt{H5Rget\_file\_name}) is supported.\\ \hline

\end{tabularx}

\textbf{Currently unsupported API calls}
\vspace{.2in} \\

\begin{tabularx}{\linewidth}{| X | >{\RaggedRight}X |}
\hline
\rowcolor{lightgray!50}%
% Center just the header row
\multicolumn{1}{| c |}{\textbf{API call}} & \multicolumn{1}{c |}{\textbf{Notes}} \\ \hline

H5Rget\_obj\_name & \\ \hline

\end{tabularx}

\end{center}

\newpage

\subsubsection{H5T interface}

\begin{center}

\textbf{Supported API calls}
\vspace{.2in} \\

\begin{tabularx}{\linewidth}{| X | >{\RaggedRight}X |}
\hline
\rowcolor{lightgray!50}%
% Center just the header row
\multicolumn{1}{| c |}{\textbf{API call}} & \multicolumn{1}{c |}{\textbf{Notes}} \\ \hline

H5Tcommit(1/2) & \\ \hline
H5Tcommit\_anon & \\ \hline
H5Topen(1/2) & \\ \hline
H5Tclose & \\ \hline
H5Tget\_create\_plist & \\ \hline
H5Tflush & \texttt{H5Tflush} is currently implemented as a no-op.\\ \hline
H5Trefresh & \texttt{H5Trefresh} is currently implemented as a no-op.\\ \hline

\end{tabularx}

\textbf{Currently unsupported API calls}
\vspace{.2in} \\

\begin{tabularx}{\linewidth}{| X | >{\RaggedRight}X |}
\hline
\rowcolor{lightgray!50}%
% Center just the header row
\multicolumn{1}{| c |}{\textbf{API call}} & \multicolumn{1}{c |}{\textbf{Notes}} \\ \hline

& \\ \hline

\end{tabularx}

\end{center}

\newpage

\subsection{Known Limitations}

The following sections outline the known current limitations of the \dvc{}.

\subsubsection{Limitations in regards to the HDF5 API}

\begin{itemize}
 \item If an application abnormally exits, the \dvc{} currently leaves the file in an unusable state. Currently, the only way to re-use the same filename after an application interruption is to use a new \acrshort{daos} pool. This issue will be resolved when rollback
to a previous snapshot is supported.
\end{itemize}

\subsubsection{Limitations in regards to DAOS}

\begin{itemize}
 \item Following the previous point about application abnormal exits, as \acrshort{daos} does not currently support forced container deletion, trying to overwrite an existing \acrshort{hdf5} file using the \texttt{H5F\_ACC\_TRUNC} flag when the file was left in an unusable state will fail; the error \textit{``can't destroy container: generic I/O error (DER\_IO)''} will be returned.
\item No support for conditional key insert/remove.
\item There is currently no support for distributed transactions.
\end{itemize}

\end{document}
